\documentclass{ctexart}

\usepackage{amsmath}
\usepackage{amssymb}
\usepackage{amsthm}
\usepackage{enumitem}
\usepackage{hyperref}

\usepackage{color, framed}
\definecolor{shadecolor}{RGB}{241, 241, 255}
\newcounter{problemname}
\newenvironment{problem}{\begin{shaded}\stepcounter{problemname}\par\noindent\textbf{题目\arabic{problemname}. }}{\end{shaded}\par}
\newenvironment{solution}{\par\noindent\textbf{解答. }}{\par}

% \usepackage{mathrsfs}

\usepackage{geometry}
\geometry{a4paper,scale=0.8}

\newcommand{\bnabla}{\overline{\nabla}}
\newcommand{\tnabla}{\widetilde{\nabla}}
\newcommand{\VecFld}{\mathscr{X}}
\newcommand{\Rm}{\mathrm{R}}
\newcommand{\bRm}{\overline{\Rm}}
\newcommand{\tRm}{\widetilde{\Rm}}

\newcommand{\Div}[2]{\nabla_{#1}{#2}}
\newcommand{\bDiv}[2]{\bnabla_{#1}{#2}}
\newcommand{\tDiv}[2]{\tnabla_{#1}{#2}}

\newcommand{\Shape}[1]{A_{#1}}
\newcommand{\Secd}{\mathbf{h}}
\newcommand{\Second}[2]{\mathbf{h}(#1, #2)}
\newcommand{\MeanCV}{\mathbf{H}}

\newcommand{\Vol}[2]{\mathrm{Vol}_{#2}(#1)}
\DeclareMathOperator{\tr}{tr}

\newcommand{\Inner}[2]{\langle {#1}, {#2} \rangle}

\def\dif{\mathop{}\!\mathrm{d}}



\title{第一次作业}
\author{洪艺中 12335025}

\begin{document}
\maketitle
\subsection{119页 习题7}
\begin{problem}
    设 $M$ 为 $C^{k + 1}$ 流形, $\nabla$ 为其上仿射联络. $X, Y$ 为 $M$ 的 $C^k$ 向量场, $X_p \not= 0$. 以 $C(s)$ 表示过 $p$ 点的向量场 $X$ 的积分曲线, $C(0) = p$. 以 $C^{-1}_{t}Y_{C(t)}$ 表示 $Y_{C(t)}$ 沿曲线 $C$ 平行移动到 $p$ 点所得的向量, 证明: 
    \[
        (\Div{X}{Y})_p = \lim_{t \rightarrow 0} \dfrac{1}{t} (C^{-1}_{t} Y_{C(t)} - Y_p).
    \]
\end{problem}
\begin{solution}
    取 $p$ 点处的一组正交标架 $\{e_1, \cdots, e_n\}$, 且 $e_1(t) = \dot{C}(t)$. 将 $e_i, i \geqslant 2$ 沿 $C(t)$ 平行移动得到 $C(t)$ 上一组标架场 $\{e_1(t), \cdots, e_n(t)\}$. 在这组基下, $Y(C(t)) = Y^i(t)e_i(t)$. 记 $Z_t(s) = C^{-1}_{s} Y_{C(t)}$ 为 $Y_{C(t)}$ 沿 $C$ 平行移动得到的向量场, 则 $Z_t(s) = Z_t^{i}(s)$ 关于 $\dot{C}(s) = e_1(s)$ 的协变导数是 $0$. 即
    \[
        \Div{e_1}{Z_t} = e_1(Z_t^{i}(s))e_i + Z_t^{i}(s)\Div{e_1}{e_i} = e_1(Z_t^{i}(s))e_i = 0
    \]
    所以 $Z_t^{i}(s)$ 都是常数, $Z_t^i(s) = Z_t^i(t) = Y^i(t)$. 因此不等式右边
    \[
        \begin{aligned}
            &\lim_{t \rightarrow 0} \dfrac{1}{t} (C^{-1}_{t} Y_{C(t)} - Y_p)
            = \lim_{t \rightarrow 0} \dfrac{1}{t} (Z_t(0) - Y_p) \\
            =& \lim_{t \rightarrow 0} \dfrac{\left(Y^i(t) - Y^i(0)\right)e_i(t)}{t} \\
            =& \dot{Y}^{i}(0)e_i(0),
        \end{aligned}
    \]
    而不等式左边 $(\Div{X}{Y})_{p} = e_1(Y^i)e_i = \dot{Y}^{i}(0)e_i(0)$. 因此两者相等.
\end{solution}

\subsection{119页 习题9}
\begin{problem}
    设 $\{e_i\}, 1 \leqslant i \leqslant m$ 为 $C^{k}$ 流形 $M^{n}$ 上的局部基向量场. $\{\omega^{i}\}$ 为其对偶基向量场, $\{\omega^i_j\}$ 为仿射联络 $1$--形式.

    \begin{enumerate}[label=(\roman*)]
        \item 证明: 测地线的微分方程为
        \[
            \dfrac{\dif}{\dif t}\left(\dfrac{\omega^{k}}{\dif t}\right) + \dfrac{\omega^{i}}{\dif t}\dfrac{\omega^{k}_{i}}{\dif t} = 0
        \]
        \item 关于 $\mathbb{R}^2$ 上直角坐标 $x^1, x^2$ 的自然标架, 设联络 $1$--形式为
        \[
            \omega^{1}_{1} = \omega^{2}_{2} = 0, \quad \omega^{1}_{2} = \dif x^{2}, \quad \omega^{2}_{1} = \dif x^{1}.
        \]
        求挠率、曲率, 并求测地线微分方程.
    \end{enumerate}
\end{problem}
\begin{solution}
    \begin{enumerate}[label=(\roman*)]
        \item 设某曲线 $\gamma(t)$ 的切向量为 $X(t)$. 则测地线方程说明
        \[
            \begin{aligned}
            0 &= \Div{X}{X} = X(\omega^{i}(X))e_i + \omega^{i}(X)\Div{X}{e_i} \\
            &= X(\omega^{k}(X))e_k + \omega^{i}(X)\omega^{k}_{i}e_k.
            \end{aligned}
        \]
        所以如果将 $X$ 写作 $\dDif{}{t}$, 则测地线方程为
        \[
            \dDif{}{t}(\omega^{k}(\dDif{}{t})) + \omega^{i}(\dDif{}{t})\omega^{k}_{i}(\dDif{}{t}) = 0.
        \]
        \item 代入给出的联络 $1$--形式, 可得
        \[
            \begin{aligned}
                \Div{e_1}{e_1} = e_2, \quad &\Div{e_1}{e_2} = 0, \\
                \Div{e_2}{e_1} = 0, \quad &\Div{e_2}{e_2} = e_1.
            \end{aligned}
        \]
        挠率: $T(e_1, e_2) = 0$, 故挠率张量 $T = 0$.
        曲率: 截面曲率 $\Rm_{1212} = \Inner{\Rm{e_1}{e_2}e_2}{e_1} = g_{12} = 0.$
        测地线方程: 设 $\gamma(t)\colon (0, 1) \rightarrow M$ 为一条曲线, $\gamma(t) = (u^1(t), u^2(t))$, 则若 $\gamma(t)$ 是测地线, 其将满足方程: 
        \begin{align*}
            \dDif{^2 x^1}{t^2} + \left(\dDif{x^2}{t}\right)^2 = 0, \\
            \dDif{^2 x^2}{t^2} + \left(\dDif{x^1}{t}\right)^2 = 0.
        \end{align*}
    \end{enumerate}
\end{solution}

\subsection{131页 习题1}
\begin{problem}
    证明由
    \[
        2\Inner{\Div{X}{Y}}{Z} = X\Inner{Y}{Z} + Y\Inner{Z}{X} - Z\Inner{X}{Y} + \Inner{[X, Y]}{Z} - \Inner{[Y, Z]}{X} + \Inner{[Z, X]}{Y}
    \]
    定义的放射联络 $\Div{X}{Y}$ 是一个黎曼联络.
\end{problem}
\begin{solution}
    需要验证
    \begin{enumerate}
        \item 这样定义的 $\Div{X}{Y}$ 关于 $X$ 是函数线性的, 而关于 $Y$ 是导数;
        \item 联络是无挠的;
        \item 联络和度量是相容的.
    \end{enumerate}
    接下来逐项验证:
    \begin{enumerate}
        \item 首先考虑
        \[
            \begin{aligned}
                2\Inner{\Div{f_1X_1 + f_2X_2}{Y}}{Z}
            \end{aligned}
        \]
    \end{enumerate}
\end{solution}
\end{document}