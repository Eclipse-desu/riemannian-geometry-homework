\documentclass{ctexart}

\usepackage{amsmath}
\usepackage{amssymb}
\usepackage{amsthm}
\usepackage{enumitem}
\usepackage{hyperref}

\usepackage{mathrsfs} % for \VecFld

\usepackage{color, framed}
\definecolor{shadecolor}{RGB}{241, 241, 255}
\newcounter{problemname}
\newenvironment{problem}{\begin{shaded}\stepcounter{problemname}\par\noindent\textbf{题目\arabic{problemname}. }}{\end{shaded}\par}
\newenvironment{solution}{\par\noindent\textbf{解答. }}{\par}

% \usepackage{mathrsfs}

\usepackage{geometry}
\geometry{a4paper,scale=0.8}

\newcommand{\bnabla}{\overline{\nabla}}
\newcommand{\tnabla}{\widetilde{\nabla}}
\newcommand{\VecFld}{\mathscr{X}}
\newcommand{\Rm}{\mathrm{R}}
\newcommand{\bRm}{\overline{\Rm}}
\newcommand{\tRm}{\widetilde{\Rm}}

\newcommand{\Div}[2]{\nabla_{#1}{#2}}
\newcommand{\bDiv}[2]{\bnabla_{#1}{#2}}
\newcommand{\tDiv}[2]{\tnabla_{#1}{#2}}

\newcommand{\Shape}[1]{A_{#1}}
\newcommand{\Secd}{\mathbf{h}}
\newcommand{\Second}[2]{\mathbf{h}(#1, #2)}
\newcommand{\MeanCV}{\mathbf{H}}

\newcommand{\Vol}[2]{\mathrm{Vol}_{#2}(#1)}
\DeclareMathOperator{\tr}{tr}

\newcommand{\Inner}[2]{\langle {#1}, {#2} \rangle}

\def\dif{\mathop{}\!\mathrm{d}}



\title{第三次作业}
\author{洪艺中 12335025}

\begin{document}
\maketitle
\subsection{132页 习题4}
\begin{problem}
    设 $M$ 为光滑流形, $\nabla$ 为对称仿射联络. 设 $\{e_i\}$ 是局部基向量场, $\{\omega^i\}$ 和 $\{\omega^{i}_{j}\}$ 分别是对偶基和联络 $1$--形式, 证明:
    \[
        \Div{X}{\omega^{i}} = -\omega^{i}_{j}(X)\omega^{j}, \quad \forall X \in \VecFld(M).
    \]
\end{problem}
\begin{solution}
    取向量场 $Y \in \VecFld(M)$, 设 $Y = Y^{i}e_i$.
    \begin{align*}
        (\Div{X}\omega^{i})(Y) 
        =& X(\omega^{i}(Y)) - \omega^{i}(\Div{X}{Y}) \\
        =& X(Y^{i}) - \omega^{i}(X(Y^{j})e_j + Y^j\omega^{k}_{j}(X)e_k) \\
        =& X(Y^{i}) - X(Y^{i}) - Y^{j}\omega^{i}_{j}(X) \\
        =& -\omega^{i}_{j}(X)\omega^j(Y).
    \end{align*}
    因此 $\Div{X}{\omega^{i}} = -\omega^{i}_{j}(X)\omega^{j}$.
\end{solution}

\subsection{132页 习题5}
\begin{problem}
    设 $(M^n, g)$ 为 Riemann 流形, $\omega = \omega_{i_1\cdots i_m}\dif x^{i_1} \wedge \cdots \wedge \dif x^{i_m}$ 为 $m$--形式, 其中
    \[
        \begin{aligned}
            \omega_{i_1 \cdots i_m} =& \sqrt{G}\varepsilon_{i_1 \cdots i_m}, \\
            \varepsilon_{i_1 \cdots i_m} =&
            \begin{cases}
                0 & (i_1, \cdots, i_m) \text{中有相同时}, \\
                1 & (i_1, \cdots, i_m) \text{为偶置换}, \\
                -1 & (i_1, \cdots, i_m) \text{为奇置换}.
            \end{cases}
        \end{aligned}
    \]
    证明: $\omega_{i_1 \cdots i_m, k} = 0$, 即 $\omega$ 是平行的.
\end{problem}
\begin{solution}
    如果按照题目, $\omega \equiv 0$, 所以此题应该是说明体积元关于 Riemann 联络是平行的.

    由于体积 $\vol = \sqrt{G}\dif x^1 \wedge \cdots \dif x^m$ 是张量, 其协变导数也是张量. 任取 $\{v_1, \cdots, v_n\} \subset \VecFld(M)$, $v_i = \alpha^{j}_{i}e_j$, 则
    % \[
    %     \Div{e_k}\vol(v_1, \cdots, v_m) = \sum_{\sigma \in S^{m}}\left(\prod_{i = 1}^{m}V^{\sigma(i)}_{i}\right)\Div{e_k}\vol(e_{\sigma{1}}, \cdots, e_{\sigma{m}}).
    % \]
    % 因此可以仅考虑 $\{v_i\}$ 都是基向量场(即 $\pdDif{}{x^{i}}$) 的情况.
    % 此时
    \[
        \begin{aligned}
            \Div{e_k}\vol(v_1, \cdots, v_m)
            =& e_k(\vol(v_1, \cdots, v_m)) - \sum_{i = 1}^{m}\vol(v_1, \cdots, \Div{e_k}{v_i}, \cdots, v_m) \\
            =& e_k(\vol(v_1, \cdots, v_m)) - \sum_{i = 1}^{m}\vol(v_1, \cdots, e_k(\alpha^{l}_{i})e_l + \alpha^{j}_{i}\omega^{l}_{j}(e_k)e_l, \cdots, v_m).
        \end{aligned}
    \]

    记矩阵 $A = (\alpha^{j}_{i})$, 以及记其伴随阵为 $A^{\star} = (A^{j}_{i})$, 满足 $\sum_{i}\alpha^{i}_{l}A^{i}_{j} = \delta_{jl}\det A$, $\sum_{i}\alpha^{j}_{i}A^{l}_{i} = \delta_{jl}\det A$. 则利用这些记号, 以及
    \[
        e_k(\det A) = \sum_{i = 1}^{m}\sum_{j = 1}^{m}e_k(\alpha^{j}_{i})A^{j}_{i},
    \]
    和
    \[
        \begin{aligned}
        e_k(\sqrt{G}) =& \frac{e_k(G)}{2\sqrt{G}} = \sqrt{G}\sum_{i, j = 1}^{m}\frac{1}{2}(\Inner{\Div{e_k}e_i}{e_j} + \Inner{e_i}{\Div{e_k}e_j})g^{ij}\\
        =& \sqrt{G}\sum_{i, j, l = 1}^{n}\omega^{l}_{i}(e_k)g_{jl}g^{ij} \\
        =& \sqrt{G}\sum_{i = 1}^{n}\omega^{i}_{i}(e_k).
        \end{aligned}
    \]
    可以得到
    \[
        \begin{aligned}
            \Div{e_k}\vol(v_1, \cdots, v_m)
            =& e_k(\vol(v_1, \cdots, v_m)) - \sum_{i = 1}^{m}\vol(v_1, \cdots, \left(e_k(\alpha^{l}_{i}) + \alpha^{j}_{i}\omega^{l}_{j}(e_k)\right)e_l, \cdots, v_m) \\
            =& e_k(\sqrt{G}\det A) - \sqrt{G}\sum_{i = 1}^{m}\sum_{l = 1}^{m}\left(e_k(\alpha^{l}_{i}) + \alpha^{j}_{i}\omega^{l}_{j}(e_k)\right) \cdot A^{l}_{i} \\
            =& e_k(\sqrt{G}\det A) - \sqrt{G}\sum_{i = 1}^{m}\sum_{l = 1}^{m}e_k(\alpha^{l}_{i})A^{l}_{i} - \sqrt{G}\sum_{l = 1}^{m}\omega^{l}_{j}(e_k)\left(\sum_{i = 1}^{m}\alpha^{j}_{i}\cdot A^{l}_{i}\right) \\
            =& e_k(\sqrt{G}\det A) - \sqrt{G}e_k(\det A) - \sqrt{G}\sum_{j, l = 1}^{m}\omega^{l}_{j}(e_k) \delta_{jl} \det A \\
            =& e_k(\sqrt{G}\det A) - \sqrt{G}e_k(\det A) - \sqrt{G}\sum_{j = 1}^{m}\omega^{j}_{j}(e_k) \det A \\
            =& e_k(\sqrt{G}\det A) - \sqrt{G}e_k(\det A) - e_k(\sqrt{G}) \det A \\
            =& 0.
        \end{aligned}
    \]
    所以 $\Div{e_k}\vol = 0$, 即体积元关于 Riemann 联络是平行的. 得证.
\end{solution}

\subsection{132页 习题8}
\begin{problem}
    设 $(M^m, g)$ 为连通的 Riemann 流形, $X_{(1)}, \cdots, X_{(r)}\ (r \leqslant m)$ 均为平行向量场. 证明:
    \begin{enumerate}[label = (\arabic* )]
        \item 如果其在 $M$ 的某一点线性无关, 则其在 $M$ 的各点均线性无关;
        \item 如果 $r = m$ 且其在某一点线性无关, 则 $(M, g)$ 的曲率张量为零向量.
    \end{enumerate}
\end{problem}
\begin{solution}
    
\end{solution}
\end{document}