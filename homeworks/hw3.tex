\documentclass{ctexart}

\usepackage{amsmath}
\usepackage{amssymb}
\usepackage{amsthm}
\usepackage{enumitem}
\usepackage{hyperref}

\usepackage{mathrsfs} % for \VecFld

\usepackage{color, framed}
\definecolor{shadecolor}{RGB}{241, 241, 255}
\newcounter{problemname}
\newenvironment{problem}{\begin{shaded}\stepcounter{problemname}\par\noindent\textbf{题目\arabic{problemname}. }}{\end{shaded}\par}
\newenvironment{solution}{\par\noindent\textbf{解答. }}{\par}

% \usepackage{mathrsfs}

\usepackage{geometry}
\geometry{a4paper,scale=0.8}

% number fields
\newcommand{\real}{\mathbb{R}}
\newcommand{\Zahl}{\mathbb{Z}}
\newcommand{\half}{\mathbb{H}}

% function spaces
\newcommand{\cont}[1]{C^{#1}}

% manifold spaces
\newcommand{\sphere}{\mathbb{S}}        % a big S

% unitary operators
\def\dif{\mathop{}\!\mathrm{d}}         % differential
\DeclareMathOperator{\tr}{tr}           % trace
\newcommand{\Inner}[2]{\langle {#1}, {#2} \rangle}            % inner product
\newcommand{\norm}[1]{\Vert {#1} \Vert}

% differential manifold
\newcommand{\VecFld}{\mathscr{X}}
\newcommand{\vol}{\mathrm{vol}}

% Riemannian connection
\newcommand{\Rm}{\mathrm{Rm}}           % curvature tensor
\newcommand{\R}{\mathrm{R}}             % curvature tensor, sometimes scalar curvature
\newcommand{\K}{\mathrm{K}}             % sectional curvature
\newcommand{\Ric}{\mathrm{Ric}}         % Ricci tensor
\newcommand{\scal}{\mathrm{scal}}       % scalar curvature

\newcommand{\Div}[1]{\nabla_{#1}}

% submanifold
\newcommand{\bnabla}{\overline{\nabla}}
\newcommand{\tnabla}{\widetilde{\nabla}}
\newcommand{\bRm}{\overline{\Rm}}
\newcommand{\tRm}{\widetilde{\Rm}}

\newcommand{\bDiv}[1]{\bnabla_{#1}}
\newcommand{\tDiv}[1]{\tnabla_{#1}}

\newcommand{\Shape}[1]{A_{#1}}
\newcommand{\Secd}{\mathbf{h}}
\newcommand{\Second}[2]{\mathbf{h}(#1, #2)}
\newcommand{\MeanCV}{\mathbf{H}}

\newcommand{\pdDif}[2]{\dfrac{\partial #1}{\partial #2}}

\newcommand{\dDif}[2]{\dfrac{\dif #1}{\dif #2}}



\title{第三次作业}
\author{洪艺中 12335025}

\begin{document}
\maketitle
\subsection{132页 习题4}
\begin{problem}
    设 $M$ 为光滑流形, $\nabla$ 为对称仿射联络. 设 $\{e_i\}$ 是局部基向量场, $\{\omega^i\}$ 和 $\{\omega^{i}_{j}\}$ 分别是对偶基和联络 $1$--形式, 证明:
    \[
        \Div{X}{\omega^{i}} = -\omega^{i}_{j}(X)\omega^{j}, \quad \forall X \in \VecFld(M).
    \]
\end{problem}
\begin{solution}
    取向量场 $Y \in \VecFld(M)$, 设 $Y = Y^{i}e_i$.
    \begin{align*}
        (\Div{X}\omega^{i})(Y) 
        =& X(\omega^{i}(Y)) - \omega^{i}(\Div{X}{Y}) \\
        =& X(Y^{i}) - \omega^{i}(X(Y^{j})e_j + Y^j\omega^{k}_{j}(X)e_k) \\
        =& X(Y^{i}) - X(Y^{i}) - Y^{j}\omega^{i}_{j}(X) \\
        =& -\omega^{i}_{j}(X)\omega^j(Y).
    \end{align*}
    因此 $\Div{X}{\omega^{i}} = -\omega^{i}_{j}(X)\omega^{j}$.
\end{solution}

\subsection{132页 习题5}
\begin{problem}
    设 $(M^n, g)$ 为 Riemann 流形, $\omega = \omega_{i_1\cdots i_m}\dif x^{i_1} \wedge \cdots \wedge \dif x^{i_m}$ 为 $m$--形式, 其中
    \[
        \begin{aligned}
            \omega_{i_1 \cdots i_m} =& \sqrt{G}\varepsilon_{i_1 \cdots i_m}, \\
            \varepsilon_{i_1 \cdots i_m} =&
            \begin{cases}
                0 & (i_1, \cdots, i_m) \text{中有相同时}, \\
                1 & (i_1, \cdots, i_m) \text{为偶置换}, \\
                -1 & (i_1, \cdots, i_m) \text{为奇置换}.
            \end{cases}
        \end{aligned}
    \]
    证明: $\omega_{i_1 \cdots i_m, k} = 0$, 即 $\omega$ 是平行的.
\end{problem}
\begin{solution}
    按照题目, $\omega$ 是体积元的常数倍数, 所以此题等价于是说明体积元关于 Riemann 联络是平行的.

    由于体积 $\vol = \sqrt{G}\dif x^1 \wedge \cdots \dif x^m$ 是张量, 其协变导数也是张量. 任取 $\{v_1, \cdots, v_n\} \subset \VecFld(M)$, $v_i = \alpha^{j}_{i}e_j$, 则
    % \[
    %     \Div{e_k}\vol(v_1, \cdots, v_m) = \sum_{\sigma \in S^{m}}\left(\prod_{i = 1}^{m}V^{\sigma(i)}_{i}\right)\Div{e_k}\vol(e_{\sigma{1}}, \cdots, e_{\sigma{m}}).
    % \]
    % 因此可以仅考虑 $\{v_i\}$ 都是基向量场(即 $\pdDif{}{x^{i}}$) 的情况.
    % 此时
    \[
        \begin{aligned}
            \Div{e_k}\vol(v_1, \cdots, v_m)
            =& e_k(\vol(v_1, \cdots, v_m)) - \sum_{i = 1}^{m}\vol(v_1, \cdots, \Div{e_k}{v_i}, \cdots, v_m) \\
            =& e_k(\vol(v_1, \cdots, v_m)) - \sum_{i = 1}^{m}\vol(v_1, \cdots, e_k(\alpha^{l}_{i})e_l + \alpha^{j}_{i}\omega^{l}_{j}(e_k)e_l, \cdots, v_m).
        \end{aligned}
    \]

    记矩阵 $A = (\alpha^{j}_{i})$, 以及记其伴随阵为 $A^{\star} = (A^{j}_{i})$, 满足 $\sum_{i}\alpha^{i}_{l}A^{i}_{j} = \delta_{jl}\det A$, $\sum_{i}\alpha^{j}_{i}A^{l}_{i} = \delta_{jl}\det A$. 则利用这些记号, 以及
    \[
        e_k(\det A) = \sum_{i, l = 1}^{m}e_k(\alpha^{l}_{i})A^{l}_{i},
    \]
    和
    \[
        \begin{aligned}
        e_k(\sqrt{G}) 
        =& \frac{e_k(G)}{2\sqrt{G}} \\
        =& \sqrt{G}\sum_{i, j = 1}^{m}\frac{1}{2}(\Inner{\Div{e_k}e_i}{e_j} + \Inner{e_i}{\Div{e_k}e_j})g^{ij}\\
        =& \sqrt{G}\sum_{i, j, l = 1}^{m}\omega^{l}_{i}(e_k)g_{jl}g^{ij} \\
        =& \sqrt{G}\sum_{i = 1}^{m}\omega^{i}_{i}(e_k).
        \end{aligned}
    \]
    可以得到
    \[
        \begin{aligned}
            \Div{e_k}\vol(v_1, \cdots, v_m)
            =& e_k(\vol(v_1, \cdots, v_m)) - \sum_{i = 1}^{m}\vol(v_1, \cdots, \left(e_k(\alpha^{l}_{i}) + \alpha^{j}_{i}\omega^{l}_{j}(e_k)\right)e_l, \cdots, v_m) \\
            =& e_k(\sqrt{G}\det A) - \sqrt{G}\sum_{i, l = 1}^{m}\left(e_k(\alpha^{l}_{i}) + \alpha^{j}_{i}\omega^{l}_{j}(e_k)\right) \cdot A^{l}_{i} \\
            =& e_k(\sqrt{G}\det A) - \sqrt{G}\sum_{i, l = 1}^{m}e_k(\alpha^{l}_{i})A^{l}_{i} - \sqrt{G}\sum_{l = 1}^{m}\omega^{l}_{j}(e_k)\left(\sum_{i = 1}^{m}\alpha^{j}_{i}\cdot A^{l}_{i}\right) \\
            =& e_k(\sqrt{G}\det A) - \sqrt{G}e_k(\det A) - \sqrt{G}\sum_{j, l = 1}^{m}\omega^{l}_{j}(e_k) \delta_{jl} \det A \\
            =& e_k(\sqrt{G}\det A) - \sqrt{G}e_k(\det A) - \sqrt{G}\sum_{j = 1}^{m}\omega^{j}_{j}(e_k) \det A \\
            =& e_k(\sqrt{G}\det A) - \sqrt{G}e_k(\det A) - e_k(\sqrt{G}) \det A \\
            =& 0.
        \end{aligned}
    \]
    所以 $\Div{e_k}\vol = 0$, 即体积元关于 Riemann 联络是平行的. 得证.
\end{solution}

\subsection{132页 习题8}
\begin{problem}
    设 $(M^m, g)$ 为连通的 Riemann 流形, $X_{(1)}, \cdots, X_{(r)}\ (r \leqslant m)$ 均为平行向量场. 证明:
    \begin{enumerate}[label = (\arabic*)]
        \item 如果这组向量在 $M$ 的某一点线性无关, 则其在 $M$ 的各点均线性无关;
        \item 如果 $r = m$ 且这组向量在某一点线性无关, 则 $(M, g)$ 的曲率张量为零张量.
    \end{enumerate}
\end{problem}
\begin{solution}
    \begin{enumerate}[label=(\arabic*)]
        \item\label{enum::1} $X_{(i)}$ 均是平行向量场, 说明任取向量场 $v$,
        \[
            \Div{v}{X_{(i)}} \equiv 0.
        \]
        那么, 反设 $\{X_{(i)}\}$ 在某点 $p$ 处线性相关, 即存在一组不全为零的系数 $c^i$,
        \[
            c^i X_{(i)}(p) = 0,
        \]
        则任取 $v \in \VecFld(M)$,
        \[
            \Div{v}{c^iX_{(i)}} = 0.
        \]
        因此在局部基表示下, $c^iX_{(i)}$ 是常向量, 而它在 $p$ 点是零向量, 所以 $c^iX_{(i)}$ 在 $M$ 上均为 $0$. 这说明 $\{X_{(i)}\}$ 在所有点线性相关. 这与条件矛盾. 结论得证.
        \item 由 \ref{enum::1} 知这组向量在 $M$ 上都线性无关, 所以 $X_{(i)}$ 形成了 $M$ 上的一组标架.
        则任取向量场 $v, w \in \VecFld(M)$, 对所有 $X_{(i)}$ 都有 $\Rm(v, w)X_{(i)} = 0$. 这相当于是说曲率张量关于某一个参数恒为 $0$, 所以曲率张量为 $0$.
    \end{enumerate}
\end{solution}

\subsection{132页 习题10}
\begin{problem}
    设 $\nabla$ 为流形 $M$ 上的对称仿射联络, $\varphi \in \mathscr{C}(T^{r}_{s}(M))$, $\psi \in \mathscr{C}(T^{p}_{q}(M))$, 则对于任意向量场 $X$, 
    \[
        \Div{X}{(\varphi \otimes \psi)} = \Div{X}\varphi \otimes \psi + \varphi \otimes \Div{X}\psi,
    \]
    特别, 若 $\theta \in A^{s}(M)$, $\omega \in A^{r}(M)$, 则有
    \[
        \Div{X}(\theta \wedge \omega) = \Div{X}\theta \wedge \omega + \theta \wedge \Div{X}\omega.
    \]
\end{problem}
\begin{solution}
    取向量场 $u_i, v_j$, $1 \leqslant i \leqslant s, 1 \leqslant j \leqslant q$. $1$--形式 $\omega^{k}, \theta^{l}$, $1 \leqslant k \leqslant r, 1 \leqslant l \leqslant p$. 为了缩减篇幅, 记 $\mathbf{U}$ 为参数组 $u_1, \cdots, u_s$, 类似地采用符号 $\mathbf{V}, \Omega, \Theta$, 
    \[
        \begin{aligned}
             &(\Div{X}{(\varphi \otimes \psi)})(\mathbf{U}, \Omega, \mathbf{V}, \Theta) \\
            =& X(\varphi(\mathbf{U}, \Omega)\psi(\mathbf{V}, \Theta)) - \sum_{i}\varphi(u_1, \cdots, \Div{X}u_i, \cdots, u_s, \Omega)\psi(\mathbf{V}, \Theta) - \sum_{k}\varphi(\mathbf{U}, \omega^{1}, \cdots, \Div{X}\omega^k, \cdots, \omega^r)\psi(\mathbf{V}, \Theta) \\
            -& \sum_{j}\varphi(\mathbf{U}, \Omega)\psi(\v_1, \cdots, \Div{X}v_j, \cdots, v_q, \Theta) - \sum_{l}\varphi(\mathbf{U}, \Omega)\psi(\mathbf{V}, \theta^1, \cdots, \Div{X}\theta^l, \cdots, \theta^p) \\
            =& (\Div{X}\varphi)(\mathbf{U}, \Omega)\psi(\mathbf{V}, \Theta) + \varphi(\mathbf{U}, \Omega)(\Div{X}\psi)(\mathbf{V}, \Theta) \\
            =& (\Div{X}\varphi \otimes \psi + \varphi \otimes \Div{X}\psi)(\mathbf{U}, \Omega, \mathbf{V}, \Theta).
        \end{aligned}
    \]
    即
    \[
        \Div{X}{(\varphi \otimes \psi)} = \Div{X}\varphi \otimes \psi + \varphi \otimes \Div{X}\psi.
    \]

    对于外积的情况, 由于 $\wedge$ 相当于是对张量(关于置换)做反对称化, 而对反对称化中求和的每项都是张量积(只不过参数的顺序改变了). 所以由上面的结论, 在每一项上 $\Div{X}$ 都满足导数的运算法则, 所以求和后也满足, 即有
    \[
        \Div{X}(\theta \wedge \omega) = \Div{X}\theta \wedge \omega + \theta \wedge \Div{X}\omega.
    \]
\end{solution}
\end{document}