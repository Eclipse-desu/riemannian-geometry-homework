\documentclass{ctexart}

\usepackage{amsmath}
\usepackage{amssymb}
\usepackage{amsthm}
\usepackage{enumitem}
\usepackage{hyperref}

\usepackage{mathrsfs} % for \VecFld

\usepackage{color, framed}
\definecolor{shadecolor}{RGB}{241, 241, 255}
\newcounter{problemname}
\newenvironment{problem}{\begin{shaded}\stepcounter{problemname}\par\noindent\textbf{题目\arabic{problemname}. }}{\end{shaded}\par}
\newenvironment{solution}{\par\noindent\textbf{解答. }}{\par}

\newcommand{\cred}{\textcolor{red}}
\newcommand{\cblue}{\textcolor{blue}}
\newcommand{\ccyan}{\textcolor{cyan}}
\newcommand{\cmag}{\textcolor{magenta}}

% \usepackage{mathrsfs}

\usepackage{geometry}
\geometry{a4paper,scale=0.8}

\newcommand{\bnabla}{\overline{\nabla}}
\newcommand{\tnabla}{\widetilde{\nabla}}
\newcommand{\VecFld}{\mathscr{X}}
\newcommand{\Rm}{\mathrm{R}}
\newcommand{\bRm}{\overline{\Rm}}
\newcommand{\tRm}{\widetilde{\Rm}}

\newcommand{\Div}[2]{\nabla_{#1}{#2}}
\newcommand{\bDiv}[2]{\bnabla_{#1}{#2}}
\newcommand{\tDiv}[2]{\tnabla_{#1}{#2}}

\newcommand{\Shape}[1]{A_{#1}}
\newcommand{\Secd}{\mathbf{h}}
\newcommand{\Second}[2]{\mathbf{h}(#1, #2)}
\newcommand{\MeanCV}{\mathbf{H}}

\newcommand{\Vol}[2]{\mathrm{Vol}_{#2}(#1)}
\DeclareMathOperator{\tr}{tr}

\newcommand{\Inner}[2]{\langle {#1}, {#2} \rangle}

\def\dif{\mathop{}\!\mathrm{d}}



\title{第五次作业}
\author{洪艺中 12335025}
\begin{document}
\maketitle
\subsection{148页 习题4}
\newcommand{\K}{\mathrm{K}}
\begin{problem}
    证明:
    \[
        \begin{aligned}
            \R(X, Y, Z, W) =& \dfrac{1}{6}[\K(X + Z, Y + W) - \K(Y + Z, X + W) - \\
            \K(X, Y + W) -& \K(Z, Y + W) - \K(X + Z, Y) - \K(X + Z, W) + \\
            \K(Y, X + W) +& \K(Z, X + W) + \K(Y + Z, W) + \K(X, W) + \\
            \K(Y + Z, W) +& \K(Z, Y) - \K(Y, W) - \K(Z, X)],
        \end{aligned}
    \]
    其中, $\K(X, Y) := \R(X, Y, X, Y)$.
\end{problem}
\begin{solution}
    将 $\K(X + Z, Y + W)$ 展开,
    \[
        \begin{aligned}
             & \K(X + Z, Y + W) \\
            =& \R(X + Z, Y + W, X + Z, Y + W) \\
            =& \R(X, Y, X, Y) + \R(X, Y, X, W) + \R(X, Y, Z, Y) + \R(X, Y, Z, W) \\
            +& \R(X, W, X, Y) + \R(X, W, X, W) + \R(X, W, Z, Y) + \R(X, W, Z, W) \\
            +& \R(Z, Y, X, Y) + \R(Z, Y, X, W) + \R(Z, Y, Z, Y) + \R(Z, Y, Z, W) \\
            +& \R(Z, W, X, Y) + \R(Z, W, X, W) + \R(Z, W, Z, Y) + \R(Z, W, Z, W) \\
            =& \K(X, Y) + \K(X, W) + \K(Z, Y) + \K(Z, W) + 2\R(X, Y, Z, W) + 2\R(X, W, Z, Y) \\
            +& 2\R(X, Y, X, W) + 2\R(X, Y, Z, Y) +2\R(X, W, Z, W) + 2\R(Z, Y, Z, W), 
        \end{aligned}
    \]
    利用
    \[
        2\R(X, Y, X, W) = \K(X, Y + W) - \K(X, Y) - \K(X, W),
    \]
    得
    \[
        \begin{aligned}
             & \K(X + Z, Y + W) \\
            =& \K(X, Y) + \K(X, W) + \K(Z, Y) + \K(Z, W) \\
            +& \K(X, Y + W) - \K(X, Y) - \K(X, W) + \K(X + Z, Y) - \K(X, Y) - \K(Z, Y) \\
            +& \K(X + Z, W) - \K(X, W) - \K(Z, W) + \K(Z, Y + W) - \K(Z, Y) - \K(Z, W) \\
            +& 2\R(X, Y, Z, W) + 2\R(X, W, Z, Y) \\
            =& \K(X, Y + W) + \K(X + Z, Y) + \K(X + Z, W) + \K(Z, Y + W) \\
            -& \K(X, Y) - \K(X, W) - \K(Z, Y) - \K(Z, W) + 2\R(X, Y, Z, W) + 2\R(X, W, Z, Y)
       \end{aligned}
    \]
    借助此式
    \[
        \begin{aligned}
             & \K(X + W, Z + Y) \\
            =& \K(X, Z + Y) + \K(X + W, Z) + \K(X + W, Y) + \K(W, Z + Y) \\
            -& \K(X, Z) - \K(X, Y) - \K(W, Z) - \K(W, Y) + 2\R(X, Z, W, Y) + 2\R(X, Y, W, Z),
        \end{aligned}
    \]
    作差, 得
    \[
        \begin{aligned}
             & \K(X + Z, Y + W) - \K(X + W, Z + Y) \\
            =& \K(X, Y + W) + \K(X + Z, Y) + \K(X + Z, W) + \K(Z, Y + W) \\
            -& \K(X, Z + Y) + \K(X + W, Z) + \K(X + W, Y) + \K(W, Z + Y) \\
            -& \K(X, W) - \K(Z, Y) + \K(X, Z) + \K(W, Y) \\
            +& 4\R(X, Y, Z, W) + 2\R(X, W, Z, Y) - 2\R(X, Z, W, Y).
        \end{aligned}
    \]

    利用第一 Bianchi 恒等式,
    \[
        \R(X, W, Z, Y) - \R(X, Z, W, Y) = -\R(X, W, Y, Z) - \R(X, Z, W, Y)  = \R(X, Y, Z, W).
    \]

    所以
    \[
        4\R(X, Y, Z, W) + 2\R(X, W, Z, Y) - 2\R(X, Z, W, Y) = 6\R(X, Y, Z, W).
    \]

    代入整理, 得
    \[
        \begin{aligned}
             & \R(X, Y, Z, W) \\
            =& \dfrac{1}{6}[\K(X + Z, Y + W) - \K(X + W, Z + Y) \\
            -& \K(X, Y + W) - \K(X + Z, Y) - \K(X + Z, W) - \K(Z, Y + W) \\
            +& \K(X, Z + Y) + \K(X + W, Z) + \K(X + W, Y) + \K(W, Z + Y) \\
            +& \K(X, W) + \K(Z, Y) - \K(X, Z) - \K(W, Y)], \\
       \end{aligned}
    \]
    得证.
\end{solution}

\subsection{148页 习题5}
\begin{problem}
    设 Riemann 流形 $(M^m, g)$ 的 Riemann 曲率张量 $R$ 满足下式
    \[
        \R(X, Y, Z, W) = \dfrac{1}{m - 1}\left\{\Ric{Y, Z}g(X, W) - \Ric(Y, W)g(X, Z)\right\},
    \]
    且 $m \geqslant 3$, 则 $(M^m, g)$ 为常曲率流形.
\end{problem}
\begin{solution}
    考虑 $p$ 点互相正交的单位向量 $X, Y \in T_pM$ 确定的截面 $\pi_{X, Y}$ 的截面曲率
    \[
        K(\pi_{X, Y}) = \R(X, Y, X, Y) = -\dfrac{1}{m - 1}\Ric(Y, Y),
    \]
    同时还有
    \[
        K(\pi_{X, Y}) = \R(Y, X, Y, X) = -\dfrac{1}{m - 1}\Ric(X, X),
    \]
    所以 $\Ric(X, X) = \Ric(Y, Y)$. 任取截面上的另一个单位向量 $Z \in \pi_{X, Y}$ 并取与其正交的 $W \in \pi_{X, Y}$, 由于截面曲率和取的向量无关, 所以 $K(\pi_{X, Y}) = -\dfrac{1}{m - 1}\Ric(Z, Z) = -\dfrac{1}{m - 1}\Ric(W, W)$. 因此在 $T_pM$ 上, 只要两个单位向量 $X, Y$ 是共面的, 那么 $\Ric(X, X) = \Ric(Y, Y)$. 而两个向量必然共面, 所以 $M$ 在 $p$ 为 Ricci 迷向的, 进而 $p$ 还为迷向点, 所以 $M$ 为迷向流形. 由 Schur 定理, 不低于三维的迷向流形是常曲率流形. 因此 $(M^m, g)$ 为常曲率流形.
\end{solution}

\subsection{148页 习题6}
\begin{problem}
    设 $(M^3, g)$ 是三维 Riemann 流形. 在任一点 $p \in M^3$, 取坐标系 $\{x^i\}$ 使得在 $p$ 点有 $g_{ij} = \Inner{\pdDif{}{x^i}}{\pdDif{}{x^j}} = 0, i \not= j$. 证明: 对互不相同的 $i, j, k$, 在 $p$ 点成立
    \begin{multline*}
        \Ric_{ij} = \dfrac{1}{g_{kk}}\R_{ikjk},
        \Ric_{ii} = \dfrac{1}{g_{jj}}\R_{ijij} + \dfrac{1}{g_{kk}}\R_{ikjk},
        \R_{ijij} - g_{ii}\Ric_{jj} - g_{jj}\Ric_{ii} + \dfrac{1}{2}\rho g_{ii}g_{jj} = 0.
    \end{multline*}
    其中 $\rho$ 是 $M^3$ 的数量曲率, 即 $\rho = g^{ij}\Ric_{ij}$.
\end{problem}
\begin{solution}
    $g^{ij} = 0, i \not= j$, 所以 $g^{ii} = \dfrac{1}{g_{ii}}$.

    对第一式, 
    \[
        \Ric_{ij} = \sum_{s, t}g^{st}\R_{isjt} = \sum_{l}g^{ll}\R_{iljl} = \dfrac{1}{g_{kk}}\R_{ikjk}.
    \]
    最后一个等号去掉求和, 是因为 $\R_{iljl}$ 只有在 $l \not= i$ 且 $l \not= j$ 时才非零, 而 $M$ 是三维的, 所以 $l$ 只能取 $k$.

    对于 $\Ric_{ii}$ 同样有 $\Ric_{ii} = \sum_{l}g^{ll}\R_{ilil} = \dfrac{1}{jj}\R_{ijij} + \dfrac{1}{k}\R_{ikik}$.

    最后一个式子, 左边为
    \[
        \begin{aligned}
             & \R_{ijij} - g_{ii}\Ric_{jj} - g_{jj}\Ric_{ii} + \dfrac{1}{2}\rho g_{ii}g_{jj} \\
            =& \R_{ijij} - g_{ii}\Ric_{jj} - g_{jj}\Ric_{ii} + \dfrac{1}{2}g_{ii}g_{jj}\sum_{l}\dfrac{\Ric_{ll}}{g_{ll}} \\
            =& \R_{ijij} - g_{ii}\Ric_{jj} - g_{jj}\Ric_{ii} + \dfrac{1}{2}g_{jj}\Ric_{ii} + \dfrac{1}{2}g_{ii}\Ric_{jj} + \dfrac{1}{2}g_{ii}g_{jj}\dfrac{\Ric_{kk}}{g_{kk}} \\
            =& \R_{ijij} - \dfrac{1}{2}g_{ii}\Ric_{jj} - \dfrac{1}{2}g_{jj}\Ric_{ii} + \dfrac{1}{2}g_{ii}g_{jj}\dfrac{\Ric_{kk}}{g_{kk}} \\
            =& \cred{\R_{ijij}} - \cred{\dfrac{1}{2}\dfrac{g_{ii}}{g_{ii}}\R_{ijij}} - \ccyan{\dfrac{1}{2}\dfrac{g_{ii}}{g_{kk}}\R_{jkjk}} - \cred{\dfrac{1}{2}\dfrac{g_{jj}}{g_{jj}}\R_{ijij}} - \cmag{\dfrac{1}{2}\dfrac{g_{jj}}{g_{kk}}\R_{ikik}} + \cmag{\dfrac{1}{2}\dfrac{g_{ii}g_{jj}}{g_{ii}g_{kk}}\R_{ikik}} + \ccyan{\dfrac{1}{2}\dfrac{g_{ii}g_{jj}}{g_{jj}g_{kk}}\R_{jkjk}} \\
            =& 0
        \end{aligned}
    \]
\end{solution}

\newpage
\subsection{148页 习题8}
\begin{problem}
    设 $M \subset \mathbb{R}^3$ 为浸入曲面, 且具有由 $\mathbb{R}^3$ 的欧氏度量诱导的 Riemann 度量, 证明 $M$ 的截面曲率即为 Gauss 曲率.
\end{problem}
\newcommand{\rvec}{\mathbf{r}}
\begin{solution}
    设 $M = \rvec(u, v)$, $u$, $v$ 是正则参数. 第一基本形式
    \[
        \dif s^2 = E(\dif u)^2 + 2F(\dif u\dif v) + G(\dif v)^2,
    \]
    其中 $E = |\rvec_u|^2$, $F = \Inner{\rvec_u}{\rvec_v}$, $G = |\rvec_v|^2$.
    
    记单位法向为 $\nu$, 第二基本形式
    \[
        -\dif \nu \cdot \dif s = L(\dif u)^2 + 2M(\dif u\dif v) + N(\dif v)^2,
    \]
    其中 $L = \Inner{\rvec_{uu}}{\nu}$, $M = \Inner{\rvec_{uv}}{\nu} = \Inner{\rvec_{vu}}{\nu}$, $N = \Inner{\rvec_{vv}}{\nu}$.

    则 Gauss 曲率为
    \[
        \kappa = \dfrac{LN - M^2}{EG - F^2}.
    \]

    接下来计算截面曲率. 记 $\mathbb{R}^3$ 上的欧氏度量为 $\overline{g}$, Riemann 联络为 $\bnabla$. $M$ 上的联络为 $\nabla$, 并记 $M \rightarrow \mathbb{R}^3$ 的浸入映射为 $\iota$. 任取 $M$ 上的切向量场 $X, Y$, 则 $g(X, Y) = \overline{g}(\iota_{\ast}X, \iota_{\ast}Y)$. 故在 $\rvec_u, \rvec_v, \nu$ 这组标架下,
    \[
        \overline{g} = \begin{pmatrix}
            E & F & \\
            F & G & \\
              &   & 1 
        \end{pmatrix}
    \]
    由于 $g$ 继承自 $\overline{g}$, 所以可以验证 $\bDiv{X}Y = \Div{X}Y + h(X, Y)\nu$, 其中 $h$ 是对称双线性映射.

    则截面曲率为
    \[
        K(\pi_{\rvec_u, \rvec_v}) = \dfrac{\R(\rvec_u, \rvec_v, \rvec_u, \rvec_v)}{\Inner{\rvec_u}{\rvec_u}\Inner{\rvec_v}{\rvec_v} - \Inner{\rvec_u}{\rvec_v}^2} = \dfrac{\R(\rvec_u, \rvec_v, \rvec_u, \rvec_v)}{EG - F^2}.
    \]
    借助 $[\rvec_u, \rvec_v] = 0$
    \[
        \R(\rvec_u, \rvec_v, \rvec_u, \rvec_v) = \Inner{\R(\rvec_u, \rvec_v)\rvec_v}{\rvec_u} = \Inner{\Div{\rvec_u}\Div{\rvec_v}\rvec_v - \Div{\rvec_v}\Div{\rvec_u}\rvec_v}{\rvec_u}.
    \]

    接下来我们要利用 $\bDiv{X}Y = \Div{X}Y + h(X, Y)\nu$, 把上式转移到 $\mathbb{R}^3$ 上去. 这也是因为在 $\bnabla$ 下, $\bDiv{\rvec_u}(X) = X_u$, $\bDiv{\rvec_v}(X) = X_v$, 便于计算. 并且由于 $g$ 是 $\overline{g}$ 的限制, 所以用 $\Inner{\cdot}{\cdot}$ 统一表示两个内积.
    \[
        \begin{aligned}
             & \R(\rvec_u, \rvec_v, \rvec_u, \rvec_v) \\
            =& \Inner{\Div{\rvec_u}\Div{\rvec_v}\rvec_v - \Div{\rvec_v}\Div{\rvec_u}\rvec_v}{\rvec_u} = \Inner{\bDiv{\rvec_u}\Div{\rvec_v}\rvec_v - \bDiv{\rvec_v}\Div{\rvec_u}\rvec_v}{\rvec_u} \\
            =& \rvec_u\left( \Inner{\Div{\rvec_v}\rvec_v}{\rvec_u} \right) - \Inner{\Div{\rvec_v}\rvec_v}{\bDiv{\rvec_u}\rvec_u} - \rvec_v\left( \Inner{\Div{\rvec_u}\rvec_v}{\rvec_u} \right) + \Inner{\Div{\rvec_u}\rvec_v}{\bDiv{\rvec_v}\rvec_u} \\
            =& \rvec_u\left( \Inner{\bDiv{\rvec_v}\rvec_v}{\rvec_u} \right) - \rvec_v\left( \Inner{\bDiv{\rvec_u}\rvec_v}{\rvec_u} \right) + \Inner{\Div{\rvec_u}\rvec_v}{\bDiv{\rvec_v}\rvec_u} - \Inner{\Div{\rvec_v}\rvec_v}{\bDiv{\rvec_u}\rvec_u} \\
            =& \cred{\Inner{\rvec_{vvu}}{\rvec_u}} + \ccyan{\Inner{\rvec_{vv}}{\rvec_{uu}}} - \cred{\Inner{\rvec_{uvv}}{\rvec_u}} - \cblue{\Inner{\rvec_{uv}}{\rvec_{uv}}} \\
            +& \cblue{\Inner{\bDiv{\rvec_u}\rvec_v}{\bDiv{\rvec_v}\rvec_u}} - \Inner{\Inner{\bDiv{\rvec_u}\rvec_v}{\nu}\nu}{\bDiv{\rvec_v}\rvec_u} - \ccyan{\Inner{\bDiv{\rvec_v}\rvec_v}{\bDiv{\rvec_u}\rvec_u}} + \Inner{\Inner{\bDiv{\rvec_v}\rvec_v}{\nu}\nu}{\bDiv{\rvec_u}\rvec_u} \\
            =& \Inner{\Inner{\bDiv{\rvec_v}\rvec_v}{\nu}\nu}{\Inner{\bDiv{\rvec_u}\rvec_u}{\nu}\nu} - \Inner{\Inner{\bDiv{\rvec_u}\rvec_v}{\nu}\nu}{\Inner{\bDiv{\rvec_v}\rvec_u}{\nu}\nu} \\
            =& \Inner{\rvec_{vv}}{\nu}\Inner{\rvec_{uu}}{\nu} - \Inner{\rvec_{uv}}{\nu}\Inner{\rvec_{vu}}{\nu} \\
            =& LN - M^2.
        \end{aligned}
    \]

    即截面曲率 $K(\pi_{\rvec_u, \rvec_v}) = \dfrac{LN - M^2}{EG - F^2} = \kappa$ 等于 Gauss 曲率.
\end{solution}

\subsection{148页 习题9}
\begin{problem}
    计算球面 $S^m(r) := \left\{ x \in \mathbb{R}^{m + 1} | \sum_{i}(x^i)^2 = r^2 \right\}$ 的截面曲率, Ricci 曲率与数量曲率. $S^m(r)$ 上的 Riemann 度量是由 $\mathbb{R}^{m + 1}$ 的欧氏度量所诱导的.
\end{problem}
\begin{solution}
    在某点 $p$ 处, 两个切向量张成的截面是一个切平面 $\pi_p$. $\pi_p$ 上的截面曲率即是二维曲面 $\exp_p(\pi_p)$ 的 Gauss 曲率, 而 $\exp_p(\pi_p)$ 是半径为 $r$ 的球面. 所以 $K(\pi_p) = \dfrac{1}{r^2}$.

    故截面曲率是 $\dfrac{1}{r^2}$, Ricci 曲率是 $\dfrac{m - 1}{r^2}$, 数量曲率是 $\dfrac{m(m - 1)}{r^2}$.
\end{solution}
\end{document}