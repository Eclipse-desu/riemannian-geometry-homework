\documentclass{ctexart}

\usepackage{amsmath}
\usepackage{amssymb}
\usepackage{amsthm}
\usepackage{enumitem}
\usepackage{hyperref}

\usepackage{mathrsfs} % for \VecFld

\usepackage{color, framed}
\definecolor{shadecolor}{RGB}{241, 241, 255}
\newcounter{problemname}
\newenvironment{problem}{\begin{shaded}\stepcounter{problemname}\par\noindent\textbf{题目\arabic{problemname}. }}{\end{shaded}\par}
\newenvironment{solution}{\par\noindent\textbf{解答. }}{\par}

% \usepackage{mathrsfs}

\usepackage{geometry}
\geometry{a4paper,scale=0.8}

% number fields
\newcommand{\real}{\mathbb{R}}
\newcommand{\Zahl}{\mathbb{Z}}
\newcommand{\half}{\mathbb{H}}

% function spaces
\newcommand{\cont}[1]{C^{#1}}

% manifold spaces
\newcommand{\sphere}{\mathbb{S}}        % a big S

% unitary operators
\def\dif{\mathop{}\!\mathrm{d}}         % differential
\DeclareMathOperator{\tr}{tr}           % trace
\newcommand{\Inner}[2]{\langle {#1}, {#2} \rangle}            % inner product
\newcommand{\norm}[1]{\Vert {#1} \Vert}

% differential manifold
\newcommand{\VecFld}{\mathscr{X}}
\newcommand{\vol}{\mathrm{vol}}

% Riemannian connection
\newcommand{\Rm}{\mathrm{Rm}}           % curvature tensor
\newcommand{\R}{\mathrm{R}}             % curvature tensor, sometimes scalar curvature
\newcommand{\K}{\mathrm{K}}             % sectional curvature
\newcommand{\Ric}{\mathrm{Ric}}         % Ricci tensor
\newcommand{\scal}{\mathrm{scal}}       % scalar curvature

\newcommand{\Div}[1]{\nabla_{#1}}

% submanifold
\newcommand{\bnabla}{\overline{\nabla}}
\newcommand{\tnabla}{\widetilde{\nabla}}
\newcommand{\bRm}{\overline{\Rm}}
\newcommand{\tRm}{\widetilde{\Rm}}

\newcommand{\bDiv}[1]{\bnabla_{#1}}
\newcommand{\tDiv}[1]{\tnabla_{#1}}

\newcommand{\Shape}[1]{A_{#1}}
\newcommand{\Secd}{\mathbf{h}}
\newcommand{\Second}[2]{\mathbf{h}(#1, #2)}
\newcommand{\MeanCV}{\mathbf{H}}

\newcommand{\pdDif}[2]{\dfrac{\partial #1}{\partial #2}}

\newcommand{\dDif}[2]{\dfrac{\dif #1}{\dif #2}}



\title{第四次作业}
\author{洪艺中 12335025}
\begin{document}
\maketitle
\subsection{133页 习题11}
\begin{problem}
    设 $(M_1, g_1)$, $(M_2, g_2)$ 均为 Riemann 流形. $\nabla^{(1)}$, $\nabla^{(2)}$ 分别为它们的 Riemann 联络. $F \colon M_1 \rightarrow M_2$ 为等距微分同胚, 即 $g_1 = F^{\ast}g_2$. 证明 $F_{\ast}(\nabla^{(1)}_{X}{Y}) = \nabla^{(2)}_{F_{\ast}X}{F_{\ast}Y}$, $\forall X, Y \in \VecFld(M_1)$.
\end{problem}
\begin{solution}
    因为 $F$ 是微分同胚, 所以 $M_1$ 和 $M_2$ 是同维数流形. 因此 $F^{\ast}_p$ 是切空间 $T_pM$ 到 $T_{F(p)}M$ 的同构. 故要证明 $F_{\ast}(\nabla^{(1)}_{X}{Y}) = \nabla^{(2)}_{F_{\ast}X}{F_{\ast}Y}$, $\forall X, Y \in \VecFld(M_1)$, 只需要证明任取 $Z \in \VecFld(M_1)$,
    \[
        g_1((\nabla^{(1)}_{X}{Y}), Z) = g_2(F_{\ast}(\nabla^{(1)}_{X}{Y}), F_{\ast}Z) = g_2(\nabla^{(2)}_{F_{\ast}X}{F_{\ast}Y}, \forall X, Y \in \VecFld(M_1), F_{\ast}Z). \tag{$\ast$}\label{eq::133-11-eq1}
    \]
    而利用 Riemann 联络的唯一性构造, 联络 $\nabla$ 和度量 $\Inner{\cdot}{\cdot}$ 满足
    \[
        2\Inner{\Div{X}{Y}}{Z} = X\Inner{Y}{Z} + Y\Inner{Z}{X} - Z\Inner{X}{Y} + \Inner{[X, Y]}{Z} - \Inner{[Y, Z]}{X} + \Inner{[Z, X]}{Y},
    \]
    要证明 (\ref{eq::133-11-eq1}) 式, 只需要证明: 任取 $X, Y, Z \in \VecFld(M_1)$,
    \[
        X(g_1(Y, Z)) = (F_{\ast}X)(g_2(F_{\ast}Y, F_{\ast}Z))
    \]
    和
    \[
        g_1([X, Y], Z) = g_2([F_{\ast}X, F_{\ast}Y], F_{\ast}Z).
    \]
    利用 $g_1 = F^{\ast}g_2$ 和 Lie 括号与切映射交换, 计算可得:
    \[
        (F_{\ast}X)(g_2(F_{\ast}Y, F_{\ast}Z))
        = X\left(g_2(F_{\ast}Y, F_{\ast}Z) \circ F\right) 
        = X((F^{\ast}g_2)(Y, Z)) 
        = X(g_1(Y, Z)),
    \]
    以及
    \[
        g_2([F_{\ast}X, F_{\ast}Y], F_{\ast}Z)
        = g_2(F_{\ast}[X, Y], F_{\ast}Z) 
        = (F^{\ast}g_2)([X, Y], Z)
        = g_1([X, Y], Z).
    \]
    所以题目得证.
\end{solution}

\subsection{113页 习题12}
\begin{problem}
    设 $(M^m, g)$ 为连通 Riemann 流形, $\nabla$ 为 Riemann 联络, $A$ 为二阶对称张量且 $\nabla A = 0$. 定义线性映射 $A^{\ast} \colon T_{p}M \rightarrow T_p{M}$, $\forall p \in M$ 如下: 对任意的 $X, Y \in T_p(M)$
    \[
        \Inner{A^{\ast}(X)}{Y}_p := A(X, Y)(p),
    \]
    设 $\rho_i$ 为 $A^{\ast}$ 的特征值, $\tilde{e_i}$ 为其相应的单位特征向量, 证明:
    \begin{enumerate}
        \item 所有特征值在 $M$ 上均为常数;
        \item 若 $\rho_h \not= \rho_k$, 则 $\Inner{e_h}{e_k} = 0$. 设 $\{\tilde{e}_i\}$ 为 $A^{\star}$ 的特征向量标架, 使得 $\Inner{\tilde{e}_i}{\tilde{e}_j} = \delta_{ij}$, 则 $\rho_h \not= \rho_k$ 时, 有
        \[
            \Inner{\Div{\tilde{e}_i}\tilde{e}_h}{\tilde{e}_k} = 0, \quad h, i, k = 1, \cdots, m;
        \]
        \item 设 $\rho_i$ 为 $r$ 重根, 对应特征向量为 $\tilde{e}_1, \cdots, \tilde{e}_r$, 则 $\tilde{e}_{r + 1}, \cdots, \tilde{e}_{m}$ 生成的分布 $\mathscr{D}$ 是完全可积的.
    \end{enumerate}
\end{problem}
\begin{solution}
    \begin{enumerate}
        \item 取 $X, Y \in \VecFld(M)$, 则根据 $\nabla A = 0$,
        \[
            X(A(\tilde{e}_i, Y))
            = A\left(\Div{X}\tilde{e}_i, Y\right) + A\left(\tilde{e}_i, \Div{X}{Y}\right)
        \]
        利用 $g$ 也关于联络平行,
        \[
            X(A(\tilde{e}_i, Y)) = X(\rho_i\Inner{\tilde{e}_i}{Y})
            = X(\rho_i)\Inner{\tilde{e}_i}{Y} + \rho_i\Inner{\Div{X}\tilde{e}_i}{Y} + \rho_i\Inner{\tilde{e}_i}{\Div{X}{Y}}.
        \]
        所以
        \[
            A\left(\Div{X}\tilde{e}_i, Y\right) = X(\rho_i)\Inner{\tilde{e}_i}{Y} + \rho_i\Inner{\Div{X}\tilde{e}_i}{Y},
        \]
        因为 $\tilde{e}_i$ 是单位向量, 所以 $\Inner{\Div{X}\tilde{e}_i}{\tilde{e}_i} = 0$, 因此在上式代入 $Y = \tilde{e}_i$, 得到
        \[
            \begin{aligned}
            A\left(\Div{X}\tilde{e}_i, \tilde{e}_i\right) 
            =& X(\rho_i)\Inner{\tilde{e}_i}{\tilde{e}_i} + \rho_i\Inner{\Div{X}\tilde{e}_i}{\tilde{e}_i} \\
            =& X(\rho_i),
            \end{aligned}
        \]
        而左边又有 $X(\rho_i) = X(A(\tilde{e}_i, \tilde{e}_i)) = 2A(\Div{X}\tilde{e}_i, \tilde{e}_i)$, 于是
        \[
            \frac{1}{2}X(\rho_i) = A\left(\Div{X}\tilde{e}_i, \tilde{e}_i\right) = X(\rho_i).
        \]
        所以 $X(\rho_i) \equiv 0$. 由 $X$ 任意性, $\rho_i$ 在 $M$ 上均为常数.
        
        \item 正交性: $\rho_h\Inner{\tilde{e}_h}{\tilde{e}_k} = A(\tilde{e}_h, \tilde{e}_k) = \rho_k\Inner{\tilde{e}_h}{\tilde{e}_k}$, 因为 $\rho_h \not= \rho_k$, 所以 $\Inner{\tilde{e}_h}{\tilde{e}_k} = 0$. 
        
        $\Inner{\Div{\tilde{e}_i}\tilde{e}_h}{\tilde{e}_k} = 0$ 利用内积为 $0$ 和 $\nabla A = 0$, 不妨设 $\rho_h \not= 0$:
        \[
            \begin{aligned}
            \Inner{\Div{\tilde{e}_i}\tilde{e}_h}{\tilde{e}_k} 
            =& -\Inner{\tilde{e}_h}{\Div{\tilde{e}_i}\tilde{e}_k}\\
            =& -\frac{1}{\rho_h}A(\tilde{e}_h, \Div{\tilde{e}_i}\tilde{e}_k)\\
            =& -\frac{1}{\rho_h}\tilde{e}_i(A(\tilde{e}_h, \tilde{e}_k)) + \frac{1}{\rho_h}A(\Div{\tilde{e}_i}\tilde{e}_h, \tilde{e}_k)\\
            =& \frac{1}{\rho_h}A(\Div{\tilde{e}_i}\tilde{e}_h, \tilde{e}_k) \\
            =& \frac{\rho_k}{\rho_h}\Inner{\Div{\tilde{e}_i}\tilde{e}_h}{\tilde{e}_k},
            \end{aligned}
        \]
        由于系数不为 $1$, 所以 $\Inner{\Div{\tilde{e}_i}\tilde{e}_h}{\tilde{e}_k} = 0$.
        \item 我们依然取 $\{\tilde{e}_i\}$ 为单位正交的, 因为这不影响分布的生成. 利用分布 Frobenius 定理, 分布完全可积当且仅当其对合, 即 $s, t > r$ 时 $[\tilde{e}_s, \tilde{e}_t]$ 可由 $\tilde{e}_{r + 1}, \cdots, \tilde{e}_{m}$ 表示. 
        
        设 $[\tilde{e}_s, \tilde{e}_t] = a^{p}\tilde{e}_{p}$. 则与 $\tilde{e}_1, \cdots, \tilde{e}_r$ 内积得
        \[
            \Inner{[\tilde{e}_s, \tilde{e}_t]}{\tilde{e}_i} = \sum_{p = 1}^{r}a^p\delta_{ip} = a^i.
        \]
        而根据第二问的结论
        \[
            \Inner{[\tilde{e}_s, \tilde{e}_t]}{\tilde{e}_i} = \Inner{\Div{\tilde{e}_s}\tilde{e}_t}{\tilde{e}_i} - \Inner{\Div{\tilde{e}_t}\tilde{e}_s}{\tilde{e}_i} = 0.
        \]
        所以 $a^i = 0$, 即分布是对合的. 因此 $\mathscr{D}$ 是完全可积的.
    \end{enumerate}
\end{solution}

\subsection{题目 B}
\begin{problem}
    证明 Ricci 恒等式: 假设 $\nabla$ 是对称联络, $\phi$ 是 $(r, s)$-型张量场, $X, Y \in \VecFld(M)$, 则
    \[
        \begin{aligned}
         & \nabla^{2}\phi(\theta^1, \cdots, \theta^r, X_1, \cdots, X_s; X, Y) - \nabla^{2}\phi(\theta^1, \cdots, \theta^r, X_1, \cdots, X_s; Y, X) \\
        =& -\Rm(X, Y)(\phi(\theta^1, \cdots, \theta^r, X_1, \cdots, X_s)) \\
        +& \sum_{a = 1}^{r}\phi(\theta^1, \cdots, \theta^{a - 1}, \Rm(X, Y)\theta^a, \theta^{a + 1}, \cdots, \theta^r, X_1, \cdots, X_s) \\
        +& \sum_{b = 1}^{r}\phi(\theta^1, \cdots, \theta^r, X_1, \cdots, X_{b - 1}, \Rm(X, Y)X_{b}, X_{b + 1}, \cdots, X_s)
        \end{aligned}
    \]
    其中 $\theta^i \in A^1(M)$, $X_i \in \VecFld(M)$,
    \[
        \Rm(X, Y)\psi := (\Div{X}\Div{Y} - \Div{Y}\Div{X} - \Div{[X, Y]})\psi,
    \]
    这里 $\psi$ 是任何 $(p, q)$-型张量场.
\end{problem}
\newcommand{\cred}{\textcolor{red}}
\newcommand{\cblue}{\textcolor{blue}}
\newcommand{\ccyan}{\textcolor{cyan}}
\newcommand{\cmag}{\textcolor{magenta}}
\begin{solution}
    类似张量的联络导数, 记
    \[
        \begin{aligned}
         & \Rm(X, Y)\phi(\theta^1, \cdots, \theta^r, X_1, \cdots, X_s) \\
        :=& \Rm(X, Y)(\phi(\theta^1, \cdots, \theta^r, X_1, \cdots, X_s)) \\
        -& \sum_{a = 1}^{r}\phi(\theta^1, \cdots, \theta^{a - 1}, \Rm(X, Y)\theta^a, \theta^{a + 1}, \cdots, \theta^r, X_1, \cdots, X_s) \\
        -& \sum_{b = 1}^{r}\phi(\theta^1, \cdots, \theta^r, X_1, \cdots, X_{b - 1}, \Rm(X, Y)X_{b}, X_{b + 1}, \cdots, X_s).
        \end{aligned}
    \]
    则 Ricci 恒等式为 $\nabla^2\phi(\cdots; X, Y) - \nabla^2\phi(\cdots; Y, X) = -\Rm(X, Y)\phi(\cdots)$.

    对称联络说明, $\Div{X}{Y} - \Div{Y}{X} - [X, Y] = 0$.
    1. 对函数 $f \in C^{\infty}(M)$,
    \[
        \begin{aligned}
             & \nabla^2f(X, Y) - \nabla^2f(Y, X) \\
            =& Y(X(f)) - \Div{Y}{X}(f) - X(Y(f)) + \Div{X}{Y}(f) \\
            =& Y(X(f)) - X(Y(f)) - [Y, X](f) \\
            =& -\Rm(X, Y)f = 0.
        \end{aligned}
    \]

    2. 对向量场 $Z \in \VecFld(M)$,
    \[
        \begin{aligned}
             & \nabla^2Z(\theta; X, Y) - \nabla^2Z(\theta; Y, X) \\
            =& \ccyan{Y(X(\theta(Z)))} - \cred{Y(\Div{X}\theta(Z))} - \cblue{X(\Div{Y}\theta(Z))} + \cmag{\Div{X}\Div{Y}\theta(Z)} - \ccyan{\Div{Y}X(\theta(Z))} + \cmag{\Div{\Div{Y}X}\theta(Z)} \\
            -& \ccyan{X(Y(\theta(Z)))} + \cblue{X(\Div{Y}\theta(Z))} + \cred{Y(\Div{X}\theta(Z))} - \cmag{\Div{Y}\Div{X}\theta(Z)} + \ccyan{\Div{X}Y(\theta(Z))} - \cmag{\Div{\Div{X}Y}\theta(Z)} \\
            =& \ccyan{(YX - XY - [Y, X])(\theta(Z))} + \cmag{(\Div{X}\Div{Y} - \Div{Y}\Div{X} - \Div{[X, Y]})\theta(Z)} \\
            =& -\Rm(X, Y)(Z(\theta)) + Z(\Rm(X, Y)\theta).
        \end{aligned}
    \]

    3. 对 $1$-形式 $\theta \in \Gamma(T^{\ast}M)$,
    \[
        \begin{aligned}
             & \nabla^2\theta(Z; X, Y) - \nabla^2\theta(Z; Y, X) \\
            =& \ccyan{Y(X(\theta(Z)))} - \cred{Y(\theta(\Div{X}Z))} - \cblue{X(\theta(\Div{Y}Z))} + \cmag{\theta(\Div{X}\Div{Y}Z)} - \ccyan{\Div{Y}X(\theta(Z))} + \cmag{\theta(\Div{\Div{Y}X}Z)} \\
            -& \ccyan{X(Y(\theta(Z)))} + \cblue{X(\theta(\Div{Y}Z))} + \cred{Y(\theta(\Div{X}Z))} - \cmag{\theta(\Div{Y}\Div{X}Z)} + \ccyan{\Div{X}Y(\theta(Z))} - \cmag{\theta(\Div{\Div{X}Y}Z)} \\
            =& \ccyan{(YX - XY - [Y, X])(\theta(Z))} + \cmag{\theta((\Div{X}\Div{Y} - \Div{Y}\Div{X} - \Div{[X, Y]})Z)} \\
            =& -\Rm(X, Y)(\theta(Z)) + \theta(\Rm(X, Y)Z).
        \end{aligned}
    \]

    4. 利用 $\Div{X}(\phi \otimes \psi) = \Div{X}\phi \otimes \psi + \phi \otimes \Div{X}\psi$,
    \[
        \begin{aligned}
         & \nabla^{2}(\phi \otimes \psi)(\cdots; X, Y) - \nabla^{2}(\phi \otimes \psi)(\cdots; Y, X) \\
        =& \Div{Y}(\nabla(\phi \otimes \psi))(\cdots; X) - \Div{X}(\nabla(\phi \otimes \psi))(\cdots; X) \\
        =& \Div{Y}(\Div{X}(\phi \otimes \psi)) - \Div{\Div{Y}{X}}(\phi \otimes \psi) - \Div{X}(\Div{Y}(\phi \otimes \psi)) + \Div{\Div{X}{Y}}(\phi \otimes \psi) \\
        =& \Div{Y}(\Div{X}\phi \otimes \psi) + \Div{Y}(\phi \otimes \Div{X}\psi) - \Div{\Div{Y}{X}}\phi \otimes \psi - \phi \otimes \Div{\Div{Y}{X}}\psi \\
        -& \Div{X}(\Div{Y}\phi \otimes \psi) - \Div{X}(\phi \otimes \Div{Y}\psi) + \Div{\Div{X}{Y}}\phi \otimes \psi + \phi \otimes \Div{\Div{X}{Y}}\psi \\
        =& \Div{Y}\Div{X}\phi \otimes \psi + \Div{X}\phi \otimes \Div{Y}\psi + \Div{Y}\phi \otimes \Div{X}\psi + \phi \otimes \Div{Y}\Div{X}\psi - \Div{\Div{Y}{X}}\phi \otimes \psi - \phi \otimes \Div{\Div{Y}{X}}\psi \\
        -& \Div{X}\Div{Y}\phi \otimes \psi - \Div{Y}\phi \otimes \Div{X}\psi - \Div{X}\phi \otimes \Div{Y}\psi - \phi \otimes \Div{X}\Div{Y}\psi + \Div{\Div{X}{Y}}\phi \otimes \psi + \phi \otimes \Div{\Div{X}{Y}}\psi \\
        =& (\Div{Y}\Div{X} - \Div{X}\Div{Y} - \Div{[Y, X]})\phi \otimes \psi + \phi \otimes (\Div{Y}\Div{X} - \Div{X}\Div{Y} - \Div{[Y, X]})\psi \\
        =& \nabla^2\phi(\cdots; X, Y) \otimes \psi(\cdots) + \phi(\cdots) \otimes \nabla^2\psi(\cdots; X, Y).
        \end{aligned}
    \]

    5. 最后利用归纳法, 假设 $r + s < k$ 时, $(r, s)$-型张量场都满足 Ricci 恒等式. 则对于 $r + s = k$ 的张量场 $\phi$, 其总可以分解为若干 $(r, s)$ 型张量单项式(即由 $X_i$ 和 $\theta^j$ 张量得到的). 则我们只需证明 Ricci 恒等式对单项式成立. 对某单项式, 其可以表示为两个指标和低于 $k$ 的张量的张量积 $\phi \otimes \psi$. 则根据第 4 部分和归纳假设,
    
    \[
        \begin{aligned}
         & \nabla^{2}(\phi \otimes \psi)(\cdots; X, Y) - \nabla^{2}(\phi \otimes \psi)(\cdots; Y, X) \\
        =& \nabla^2\phi(\cdots; X, Y) \otimes \psi(\cdots) + \phi(\cdots) \otimes \nabla^2\psi(\cdots; X, Y) \\
        =& -\Rm(X, Y)\phi(\cdots) \otimes \psi(\cdots) - \phi(\cdots) \otimes \Rm(X, Y)\psi(\cdots) \\
        =& -\Rm(X, Y)(\phi(\cdots))\psi(\cdots) + \sum_{i}\phi(\cdots, \Rm(X, Y)u_i, \cdots)\psi(\cdots) \\
        -& \phi(\cdots)\Rm(X, Y)\psi(\cdots) + \phi(\cdots)\sum_{j}\psi(\cdots, \Rm(X, Y)v_j, \cdots) \\
        =& -\Rm(X, Y)(\phi \otimes \psi(\cdots)) + \sum_{k}\phi \otimes \psi(\cdots, \Rm(X, Y)w_k, \cdots) \\
        =& -\Rm(X, Y)(\phi \otimes \psi)(\cdots).
        \end{aligned}
    \]
    因此 Ricci 恒等式对任意张量成立.
\end{solution}

\subsection{习题 C}
\begin{problem}
    证明局部标架的 Ricci 恒等式
    \[
        \phi_{j_1 \cdots j_s, kl}^{i_1 \cdots i_r} - \phi_{j_1 \cdots j_s, lk}^{i_1 \cdots i_r}
        = \sum_{\alpha = 1}^{s}\phi_{j_1 \cdots j_{\alpha - 1} h j_{\alpha + 1} \cdots j_s}^{i_1 \cdots i_r}\R^{h}_{j_{\alpha}kl}
        - \sum_{\beta = 1}^{s}\phi_{j_1 \cdots j_s}^{i_1 \cdots i_{\beta - 1} h i_{\beta + 1}  \cdots i_r}\R^{i_{\beta}}_{hkl}.
    \]
\end{problem}
\begin{solution}
    首先
    \[
        \Rm(e_k, e_l)e_j = \R^h_{jkl}e_h,
    \]
    以及
    \[
        \Rm(e_k,e_l)\omega^i(e_s) = -\omega^i(\Rm(e_k, e_l)e_s) = -\omega^i(\R^h_{skl}e_h) = -\R^i_{skl}.
    \]
    所以
    \[
        \Rm(e_k,e_l)\omega^i = -\R^i_{hkl}\omega^h.
    \]
    因此利用习题 B 证明的 Ricci 恒等式,
    \[
        \begin{aligned}
         & \phi_{j_1 \cdots j_s, kl}^{i_1 \cdots i_r} - \phi_{j_1 \cdots j_s, lk}^{i_1 \cdots i_r} \\
        =& \nabla^{2}\phi(\omega^1, \cdots, \omega^r, e_1, \cdots, e_s; e_k, e_l) - \nabla^{2}\phi(\omega^1, \cdots, \omega^r, e_1, \cdots, e_s; e_l, e_k) \\
        =& -\Rm(e_k, e_l)(\phi(\omega^1, \cdots, \omega^r, e_1, \cdots, e_s)) \\
        +& \sum_{\alpha = 1}^{r}\phi(\omega^1, \cdots, \omega^{\alpha - 1}, \Rm(e_k, e_l)\omega^\alpha, \omega^{\alpha + 1}, \cdots, \omega^r, e_1, \cdots, e_s) \\
        +& \sum_{b = 1}^{r}\phi(\omega^1, \cdots, \omega^r, e_1, \cdots, e_{\beta - 1}, \Rm(e_k, e_l)e_{\beta}, e_{\beta + 1}, \cdots, e_s) \\
        =& \sum_{\alpha = 1}^{s}\phi_{j_1 \cdots j_{\alpha - 1} h j_{\alpha + 1} \cdots j_s}^{i_1 \cdots i_r}\R^{h}_{j_{\alpha}kl}
        - \sum_{\beta = 1}^{s}\phi_{j_1 \cdots j_s}^{i_1 \cdots i_{\beta - 1} h i_{\beta + 1}  \cdots i_r}\R^{i_{\beta}}_{hkl}.
        \end{aligned}
    \]
\end{solution}
\end{document}