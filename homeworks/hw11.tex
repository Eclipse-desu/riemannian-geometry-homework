\documentclass{ctexart}

\usepackage{amsmath}
\usepackage{amssymb}
\usepackage{amsthm}
\usepackage{enumitem}
\usepackage{hyperref}

\usepackage{mathrsfs} % for \VecFld
\usepackage{stmaryrd} % for \owedge

\usepackage{xcolor, framed}
\definecolor{shadecolor}{RGB}{241, 241, 255}
\newcounter{problemname}
\newenvironment{problem}{\begin{shaded}\stepcounter{problemname}\par\noindent\textbf{题目\arabic{problemname}. }}{\end{shaded}\par}
\newenvironment{solution}{\par\noindent\textbf{解答. }}{\par}

\newtheorem*{lemma*}{引理}

\newcommand{\cred}{\textcolor{red}}
\newcommand{\cblue}{\textcolor{blue}}
\newcommand{\ccyan}{\textcolor{cyan}}
\newcommand{\cmag}{\textcolor{magenta}}
\newcommand{\cpink}{\textcolor{pink}}

\usepackage{bbm}

\usepackage{geometry}
\geometry{a4paper,scale=0.8}

% number fields
\newcommand{\real}{\mathbb{R}}
\newcommand{\Zahl}{\mathbb{Z}}
\newcommand{\half}{\mathbb{H}}

% function spaces
\newcommand{\cont}[1]{C^{#1}}

% manifold spaces
\newcommand{\sphere}{\mathbb{S}}        % a big S

% unitary operators
\def\dif{\mathop{}\!\mathrm{d}}         % differential
\DeclareMathOperator{\tr}{tr}           % trace
\newcommand{\Inner}[2]{\langle {#1}, {#2} \rangle}            % inner product
\newcommand{\norm}[1]{\Vert {#1} \Vert}

% differential manifold
\newcommand{\VecFld}{\mathscr{X}}
\newcommand{\vol}{\mathrm{vol}}

% Riemannian connection
\newcommand{\Rm}{\mathrm{Rm}}           % curvature tensor
\newcommand{\R}{\mathrm{R}}             % curvature tensor, sometimes scalar curvature
\newcommand{\K}{\mathrm{K}}             % sectional curvature
\newcommand{\Ric}{\mathrm{Ric}}         % Ricci tensor
\newcommand{\scal}{\mathrm{scal}}       % scalar curvature

\newcommand{\Div}[1]{\nabla_{#1}}

% submanifold
\newcommand{\bnabla}{\overline{\nabla}}
\newcommand{\tnabla}{\widetilde{\nabla}}
\newcommand{\bRm}{\overline{\Rm}}
\newcommand{\tRm}{\widetilde{\Rm}}

\newcommand{\bDiv}[1]{\bnabla_{#1}}
\newcommand{\tDiv}[1]{\tnabla_{#1}}

\newcommand{\Shape}[1]{A_{#1}}
\newcommand{\Secd}{\mathbf{h}}
\newcommand{\Second}[2]{\mathbf{h}(#1, #2)}
\newcommand{\MeanCV}{\mathbf{H}}

\newcommand{\pdDif}[2]{\dfrac{\partial #1}{\partial #2}}

\newcommand{\dDif}[2]{\dfrac{\dif #1}{\dif #2}}



\title{第十一次作业}
\author{洪艺中 12335025}
\begin{document}
\maketitle

\subsection{192页 习题4}

\begin{problem}
    在单位球面 $\sphere^2$ 上, 举例说明: 测地线对其上任意两点为相对最短这一结论未必成立.
\end{problem}

\begin{solution}
    只需要找一条长度超过半周长的大弧即可. 例如先从北极沿 $0^\circ$ 经线(经过格林尼治天文台方向)往南极走, 到达南极之后沿 $180^\circ$ 经线走一定距离(但不走出南极大陆), 那么这条测地线就不是极短的.
\end{solution}

\subsection{192页 习题7}

\begin{problem}
    试证定理 (4.\ 2.\ 8) 的两个推论.
\end{problem}

\begin{solution}
    \begin{enumerate}[label = \textbf{推论}\arabic*]
        \item 若 $\gamma(a)$ 和 $\gamma(b)$ 不是共轭的, 则 $\gamma$ 的 Jacobi 场由其在端点的值完全确定: 如果有两个端点处相同的 Jacobi 场, 那么两者作差, 得到一个端点为 $0$ 的 Jacobi 场. 由定理 (4.\ 2.\ 8) 这个 Jacobi 场为 $0$. 所以原来的两个 Jacobi 场相等.
        \item 指标形式具有非平凡零化空间当且仅当$\gamma(a)$ 和 $\gamma(b)$ 沿 $\gamma$ 是共轭的: 如果共轭, 存在非平凡的端点为 $0$ 的 Jacobi 场. 则因为 Jacobi 场是零化子, 所以零化空间非平凡; 如果零化空间非平凡, 那么利用零化子是 Jacobi 场, 所以存在非平凡的端点为 $0$ 的 Jacobi 场, 因此两点共轭.
    \end{enumerate}
\end{solution}

\subsection{192页 习题9}
\begin{problem}
    设 $T_{\gamma}(\Omega)$ 表示沿测地线 $\gamma \colon [0, 1] \rightarrow \mfd$ 为分段光滑, 在 $\gamma$ 端点消失, 且使 $\Inner{V}{T} = 0$ 的向量场 $V$ 构成的向量空间. 试证: 在 $T_{\gamma}(\Omega)$ 上指标形式的零化度 $\nu$ 满足 $0 \leqslant \nu \leqslant m - 1$, 这里 $m = \dim \mfd$; 并且以 $m$ 维球面 $\sphere^m$ 上的对径点为例说明 $\nu$ 可以取道最大值 $m - 1$.
\end{problem}

\begin{solution}
    因为零化向量场是和测地线垂直的 Jacobi 场, 因此初值 $\Div{T}J(0)$ 可行空间最多是 $\dim \mfd - 1 = m - 1$ 维的. 因此 $0 \leqslant \nu \leqslant m - 1$.

    在 $\sphere^m$ 上, 取 $p, q$ 是对径点, 以及 $\gamma(t) = \exp_p tu$, $\gamma(\pi) = q$. 则在 $T_p\mfd$ 中, 有 $m - 1$ 个单位向量 $e_1, \cdots, e_{m - 1}$ 与 $u$ 正交, 则初值为 $J_i(0) = 0$, $\Div{T}J_i(0) = e_i$ 的 $m - 1$ 个 Jacobi 场张成了 $m - 1$ 维的零化空间.
\end{solution}

\subsection{问题 1.2}

\begin{problem}
    假设 $\gamma \colon [0, \ell] \rightarrow \mfd$ 是测地线, $p = \gamma(0)$, $v = \dot{\gamma}(0)$. 给定 $w \in T_p \mfd$, $w \not= 0$, 考虑向量场
    \[
        J(t) = (\dif \exp_p)_{tv} tw,
    \]
    证明该向量场满足: 
    \[
        |J(t)|^2 = |w|^2 t^2 - \dfrac{1}{3}\K(\Pi_p)(|v|^2|w|^2 - \Inner{v}{w}^2)t^4 + o(t^4),
    \]
    其中 $\Pi_p$ 是 $T_p \mfd$ 中由 $v$ 和 $w$ 张成的 $2$ 维截面, $\K(\Pi_p)$ 表示截面曲率.
\end{problem}

\begin{solution}
    为方便记 $f(t) = \Inner{J(t)}{J(t)}$,
    \[
        \begin{aligned}
            f(0) ={} & \Inner{J(t)}{J(t)} 
            ={} 0; \\
            f^\prime(0) ={} & 2\Inner{\Div{T}J(0)}{J(0)} 
            ={}  0; \\
            f^{\prime\prime}(0) 
            ={} & 2\Inner{\Div{T}J(0)}{\Div{T}J(0)} + 2\Inner{\Div{T}\Div{T}J(0)}{J(0)} \\
            ={} & 2|w|^2; \\
            f^{\prime\prime\prime}(0) 
            ={} & 6\Inner{\Div{T}\Div{T}J(0)}{\Div{T}J(0)} + 2\Inner{\Div{T}\Div{T}\Div{T}J(0)}{J(0)} \\
            ={} & 4\Inner{\Rm(T, J(0))T}{\Div{T}J(0)} = 0; \\
            f^{\prime\prime\prime\prime}(0) ={} & 8\Div{T}(\R(\Div{T}J, T, T, J))(0) + 6\Inner{\Div{T}\Div{T}J(0)}{\Div{T}\Div{T}J(0)} \\
            ={} & -8\R(T, \Div{T}J(0), T, \Div{T}J(0)) + 6|\Rm(T, J(0))T|^2 \\
            ={} & -8\K(\Pi_p)(|v|^2|w|^2 - \Inner{v}{w}^2).
        \end{aligned}
    \]
    所以用 Taylor 公式得到
    \[
        |J(t)|^2 = |w|^2 t^2 - \dfrac{1}{3}\K(\Pi_p)(|v|^2|w|^2 - \Inner{v}{w}^2)t^4 + o(t^4).
    \]
\end{solution}

\end{document}