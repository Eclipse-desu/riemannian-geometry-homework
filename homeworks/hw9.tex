\documentclass{ctexart}

\usepackage{amsmath}
\usepackage{amssymb}
\usepackage{amsthm}
\usepackage{enumitem}
\usepackage{hyperref}

\usepackage{mathrsfs} % for \VecFld
\usepackage{stmaryrd} % for \owedge

\usepackage{xcolor, framed}
\definecolor{shadecolor}{RGB}{241, 241, 255}
\newcounter{problemname}
\newenvironment{problem}{\begin{shaded}\stepcounter{problemname}\par\noindent\textbf{题目\arabic{problemname}. }}{\end{shaded}\par}
\newenvironment{solution}{\par\noindent\textbf{解答. }}{\par}

\newcommand{\cred}{\textcolor{red}}
\newcommand{\cblue}{\textcolor{blue}}
\newcommand{\ccyan}{\textcolor{cyan}}
\newcommand{\cmag}{\textcolor{magenta}}
\newcommand{\cpink}{\textcolor{pink}}

\usepackage{bbm}

\usepackage{geometry}
\geometry{a4paper,scale=0.8}

\newcommand{\bnabla}{\overline{\nabla}}
\newcommand{\tnabla}{\widetilde{\nabla}}
\newcommand{\VecFld}{\mathscr{X}}
\newcommand{\Rm}{\mathrm{R}}
\newcommand{\bRm}{\overline{\Rm}}
\newcommand{\tRm}{\widetilde{\Rm}}

\newcommand{\Div}[2]{\nabla_{#1}{#2}}
\newcommand{\bDiv}[2]{\bnabla_{#1}{#2}}
\newcommand{\tDiv}[2]{\tnabla_{#1}{#2}}

\newcommand{\Shape}[1]{A_{#1}}
\newcommand{\Secd}{\mathbf{h}}
\newcommand{\Second}[2]{\mathbf{h}(#1, #2)}
\newcommand{\MeanCV}{\mathbf{H}}

\newcommand{\Vol}[2]{\mathrm{Vol}_{#2}(#1)}
\DeclareMathOperator{\tr}{tr}

\newcommand{\Inner}[2]{\langle {#1}, {#2} \rangle}

\def\dif{\mathop{}\!\mathrm{d}}



\title{第九次作业}
\author{洪艺中 12335025}
\begin{document}
\maketitle

\subsection{174页 习题4}

\newcommand{\dx}[1]{ \pdDif{}{x^{#1}}}
\begin{problem}
    设 $(M, g)$ 为 Riemann 流形. $(U, \varphi, x^i)$ 是以 $q$ 为原点的法坐标图.
    \[
        X_0 = \xi^i \bigl( \dx{i} \bigr)_{q}, \quad Y_0 = \eta^i \bigl( \dx{i} \bigr)_{q}
    \]
    均为单位向量. $C \colon [0, r) \rightarrow C(s)$ 为在 $q = C(0)$ 点以 $X_0$ 为切向量的测地线, $Y(s)$ 是将 $Y_0$ 沿 $C$ 平行移动而得的切向量. 证明:
    \begin{enumerate}[label = (\roman*)]
        \item 在法坐标系的原点
        \[
            \pdDif{\Gamma^k_{ij}}{x^l} =  -\dfrac{1}{3}(\R^k_{ijl} + \R^{k}_{jil}),
        \]
        \item 设 $Y(s) = \zeta^i \bigl( \dx{i} \bigr)_{C(s)}$, 则
        \[
            \zeta^i(s) = \eta^i + \dfrac{1}{6}(\R^i_{jkl})_q \xi^j \eta^k \xi^l s^2 + o(s^3),
        \]
        \item 若 $\Inner{X_0}{Y_0} = 0$, 且令 $||Y(s)||_q^2 = g_{ij}(q)\xi^i(s) \xi^j(s)$, 则
        \[
            ||Y(s)||_q = 1 + \dfrac{s^2}{6}\R(X_0, Y_0, X_0, Y_0) + o(s^3).
        \]
    \end{enumerate}
\end{problem}

\begin{solution}
    记 $e_i = \dx{i}$.
    \begin{enumerate}[label = (\roman*)]
        \item\label{enum::1} 记 $M$ 的维数为 $n$. 任取 $\mathbf{u} = (u^1, \cdots, u^n) \in T_q M$, 因为 $\exp_q(t\mathbf{u}) = (tu^1, \cdots, tu^n)$ 是测地线, 所以由测地线方程,
        \[
            \Gamma^{k}_{ij}(\exp_q(t\mathbf{u})) u^i u^j = 0.
        \]
        因此在这一点, Riemann 曲率张量为
        \[
            \begin{aligned}
            \R^{i}_{jkl}(q) ={}& g^{im}\Inner{\Rm(e_k, e_l)e_j}{e_m} \\
            ={}& \pdDif{\Gamma^{i}_{jl}}{x^k} + \Gamma^{i}_{ks}\Gamma^{s}_{jl} - \pdDif{\Gamma^{i}_{jk}}{x^l} - \Gamma^{i}_{ls}\Gamma^{s}_{jk} \\
            ={}& \pdDif{\Gamma^{i}_{jl}}{x^k}- \pdDif{\Gamma^{i}_{jk}}{x^l}.
            \end{aligned}
        \]
        由 $\mathrm{u}$ 的任意性, $\Gamma^{k}_{ij}(q) = 0$. 对测地线方程关于 $t$ 微分, 得到
        \[
            \mathbf{u}(\Gamma^{k}_{ij}) u^i u^j = \pdDif{\Gamma^k_{ij}}{x^l}u^i u^j u^l = 0.
        \]
        取 $\mathbf{u} = u^i e_i$, 那么
        \[
            \pdDif{\Gamma^k_{ii}}{x^i}u^i u^i u^i = 0,
        \]
        取 $\mathbf{u} = e_i + e_j$ 和 $\mathbf{u} = e_i - e_j$, 那么
        \[
            \pdDif{\Gamma^{k}_{ii}}{x^j} + 2\pdDif{\Gamma^{k}_{ij}}{x^i} + 2\pdDif{\Gamma^{k}_{ij}}{x^j} + \pdDif{\Gamma^{k}_{jj}}{x^i} = 0,
        \]
        \[
            -\pdDif{\Gamma^{k}_{ii}}{x^j} - 2\pdDif{\Gamma^{k}_{ij}}{x^i} + 2\pdDif{\Gamma^{k}_{ij}}{x^j} + \pdDif{\Gamma^{k}_{jj}}{x^i} = 0,
        \]
        所以 
        \[
            \begin{aligned}
                0 ={} & \pdDif{\Gamma^{k}_{ii}}{x^j} + 2\pdDif{\Gamma^{k}_{ij}}{x^i} \\
                ={} & 3\pdDif{\Gamma^{k}_{ij}}{x^i} + \R^k_{iji} \\
                ={} & 3\pdDif{\Gamma^{k}_{ii}}{x^j} + 2\R^k_{iij}
            \end{aligned}
        \]
        因此 
        \[
            \pdDif{\Gamma^{k}_{ij}}{x^i} = -\dfrac{1}{3}\R^k_{iji} = -\dfrac{1}{3}(\R^k_{iji} + \R^{jii}),
        \]
        \[
            \pdDif{\Gamma^{k}_{ii}}{x^j} = -\dfrac{2}{3}\R^k_{iij}.
        \]

        取 $\mathbf{u} = u^i e_i + u^j e_j + u^k e_k$, 由上面的计算, 如果求和中 $i, j, k$ 只选到一个或两个下标, 那么这部分求和项为 $0$. 所以
        \[
            \begin{aligned}
                0 ={} & \pdDif{\Gamma^{k}_{ij}}{x^l} + \pdDif{\Gamma^{k}_{jl}}{x^i} + \pdDif{\Gamma^{k}_{il}}{x^j} \\
                ={} & 3\pdDif{\Gamma^{k}_{ij}}{x^l} + \bigl( \pdDif{\Gamma^{k}_{jl}}{x^i} - \pdDif{\Gamma^{k}_{ij}}{x^l} \bigr) + \bigl( \pdDif{\Gamma^{k}_{il}}{x^j} - \pdDif{\Gamma^{k}_{ij}}{x^l} \bigr) \\
                ={} & 3\pdDif{\Gamma^{k}_{ij}}{x^l} + \R^{k}_{jil} + \R^{k}_{ijl}.
            \end{aligned}
        \]
        因此
        \[
            \pdDif{\Gamma^{k}_{ij}}{x^l} = -\dfrac{1}{3}(\R^{k}_{jil} + \R^{k}_{ijl}).
        \]
        \item $Y(s)$ 满足平行移动方程
        \[
            \pdDif{\zeta^i(s)}{s} + \Gamma^{i}_{jk} \zeta^j(s) \xi^k(s) = 0,
        \]
        所以在 $q$ 点
        \[
            \pdDif{\zeta^i(s)}{s} = -\Gamma^{i}_{jk} \eta^j \xi^k = 0,
        \]
        并且利用 $\Gamma^i_{jk}(q) = 0$, 在 $q$ 点二阶导数为
        \[
            \pdDif{^2 \zeta^i(s)}{s^2} = -\pdDif{\Gamma^{i}_{jk}}{s}(q) \eta^j \xi^k.
        \]
        利用 \ref{enum::1} 的结论,
        \[
            \pdDif{\Gamma^{k}_{ij}}{s}(q) = -\dfrac{1}{3}(\R^{k}_{jil} + \R^{k}_{ijl})\xi^l,
        \]
        所以
        \[
            \begin{aligned}
                \zeta^i(s) ={} & \eta^i + \pdDif{\zeta^i(s)}{s} s + \dfrac{1}{2}\pdDif{^2 \zeta^i(s)}{s^2} s^2 + O(s^3) \\
                ={} & \eta^i + \dfrac{1}{6}(\R^i_{jkl} + \R^i_{kjl}) \eta^j \xi^k \xi^l s^2 + O(s^3).
            \end{aligned}
        \]
        因为 $\R^i_{jkl}$ 交换 $kl$ 会变号, 所以 $\R^i_{jkl}\eta^j \xi^k \xi^l = 0$. 因此
        \[
            \zeta^i(s) = \eta^i + \dfrac{1}{6}\R^i_{kjl} \xi^j \eta^k \xi^l s^2 + O(s^3).
        \]
        \item $\Inner{X_0}{Y_0} = 0$, 则 $\sum_{i} \eta^i \xi^i = 0$.
        所以
        \[
            \begin{aligned}
                ||Y(s)||^2_q ={} & g_{ij}(q)\xi^i(s) \xi^j(s) \\
                ={} & |\eta^i|^2 + \sum_{i}\dfrac{1}{6}\R_{ijkl} \eta^i \xi^j \eta^k \xi^l s^2 + O(s^3) \\
                ={} & 1 + \dfrac{1}{6}R(X_0, Y_0, X_0, Y_0)s^2 + O(s^3).
            \end{aligned}
        \]
    \end{enumerate}
\end{solution}

\end{document}