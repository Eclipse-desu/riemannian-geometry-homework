\documentclass{ctexart}

\usepackage{amsmath}
\usepackage{amssymb}
\usepackage{amsthm}
\usepackage{enumitem}
\usepackage{hyperref}

\usepackage{mathrsfs} % for \VecFld
\usepackage{stmaryrd} % for \owedge

\usepackage{xcolor, framed}
\definecolor{shadecolor}{RGB}{241, 241, 255}
\newcounter{problemname}
\newenvironment{problem}{\begin{shaded}\stepcounter{problemname}\par\noindent\textbf{题目\arabic{problemname}. }}{\end{shaded}\par}
\newenvironment{solution}{\par\noindent\textbf{解答. }}{\par}

\newtheorem*{lemma*}{引理}

\newcommand{\cred}{\textcolor{red}}
\newcommand{\cblue}{\textcolor{blue}}
\newcommand{\ccyan}{\textcolor{cyan}}
\newcommand{\cmag}{\textcolor{magenta}}
\newcommand{\cpink}{\textcolor{pink}}

\usepackage{bbm}

\usepackage{geometry}
\geometry{a4paper,scale=0.8}

\newcommand{\bnabla}{\overline{\nabla}}
\newcommand{\tnabla}{\widetilde{\nabla}}
\newcommand{\VecFld}{\mathscr{X}}
\newcommand{\Rm}{\mathrm{R}}
\newcommand{\bRm}{\overline{\Rm}}
\newcommand{\tRm}{\widetilde{\Rm}}

\newcommand{\Div}[2]{\nabla_{#1}{#2}}
\newcommand{\bDiv}[2]{\bnabla_{#1}{#2}}
\newcommand{\tDiv}[2]{\tnabla_{#1}{#2}}

\newcommand{\Shape}[1]{A_{#1}}
\newcommand{\Secd}{\mathbf{h}}
\newcommand{\Second}[2]{\mathbf{h}(#1, #2)}
\newcommand{\MeanCV}{\mathbf{H}}

\newcommand{\Vol}[2]{\mathrm{Vol}_{#2}(#1)}
\DeclareMathOperator{\tr}{tr}

\newcommand{\Inner}[2]{\langle {#1}, {#2} \rangle}

\def\dif{\mathop{}\!\mathrm{d}}



\title{第九次作业}
\author{洪艺中 12335025}
\begin{document}
\maketitle

\subsection{174页 习题4}\label{sec::1}

\newcommand{\dx}[1]{ \pdDif{}{x^{#1}}}
\begin{problem}
    设 $(M, g)$ 为 Riemann 流形. $(U, \varphi, x^i)$ 是以 $q$ 为原点的法坐标图.
    \[
        X_0 = \xi^i \bigl( \dx{i} \bigr)_{q}, \quad Y_0 = \eta^i \bigl( \dx{i} \bigr)_{q}
    \]
    均为单位向量. $C \colon [0, r) \rightarrow C(s)$ 为在 $q = C(0)$ 点以 $X_0$ 为切向量的测地线, $Y(s)$ 是将 $Y_0$ 沿 $C$ 平行移动而得的切向量. 证明:
    \begin{enumerate}[label = (\roman*)]
        \item 在法坐标系的原点
        \[
            \pdDif{\Gamma^k_{ij}}{x^l} =  -\dfrac{1}{3}(\R^k_{ijl} + \R^{k}_{jil}),
        \]
        \item 设 $Y(s) = \zeta^i \bigl( \dx{i} \bigr)_{C(s)}$, 则
        \[
            \zeta^i(s) = \eta^i + \dfrac{1}{6}(\R^i_{jkl})_q \xi^j \eta^k \xi^l s^2 + o(s^3),
        \]
        \item 若 $\Inner{X_0}{Y_0} = 0$, 且令 $||Y(s)||_q^2 = g_{ij}(q)\xi^i(s) \xi^j(s)$, 则
        \[
            ||Y(s)||_q = 1 + \dfrac{s^2}{6}\R(X_0, Y_0, X_0, Y_0) + o(s^3).
        \]
    \end{enumerate}
\end{problem}

\begin{solution}
    记 $e_i = \dx{i}$.
    \begin{enumerate}[label = (\roman*)]
        \item\label{enum::1} 记 $M$ 的维数为 $n$. 任取 $\mathbf{u} = (u^1, \cdots, u^n) \in T_q M$, 因为 $\exp_q(t\mathbf{u}) = (tu^1, \cdots, tu^n)$ 是测地线, 所以由测地线方程,
        \[
            \Gamma^{k}_{ij}(\exp_q(t\mathbf{u})) u^i u^j = 0.
        \]
        因此在这一点, Riemann 曲率张量为
        \[
            \begin{aligned}
            \R^{i}_{jkl}(q) ={}& g^{im}\Inner{\Rm(e_k, e_l)e_j}{e_m} \\
            ={}& \pdDif{\Gamma^{i}_{jl}}{x^k} + \Gamma^{i}_{ks}\Gamma^{s}_{jl} - \pdDif{\Gamma^{i}_{jk}}{x^l} - \Gamma^{i}_{ls}\Gamma^{s}_{jk} \\
            ={}& \pdDif{\Gamma^{i}_{jl}}{x^k}- \pdDif{\Gamma^{i}_{jk}}{x^l}.
            \end{aligned}
        \]
        由 $\mathrm{u}$ 的任意性, $\Gamma^{k}_{ij}(q) = 0$. 对测地线方程关于 $t$ 微分, 得到
        \[
            \mathbf{u}(\Gamma^{k}_{ij}) u^i u^j = \pdDif{\Gamma^k_{ij}}{x^l}u^i u^j u^l = 0.
        \]
        取 $\mathbf{u} = u^i e_i$, 那么
        \[
            \pdDif{\Gamma^k_{ii}}{x^i}u^i u^i u^i = 0,
        \]
        取 $\mathbf{u} = e_i + e_j$ 和 $\mathbf{u} = e_i - e_j$, 那么
        \[
            \pdDif{\Gamma^{k}_{ii}}{x^j} + 2\pdDif{\Gamma^{k}_{ij}}{x^i} + 2\pdDif{\Gamma^{k}_{ij}}{x^j} + \pdDif{\Gamma^{k}_{jj}}{x^i} = 0,
        \]
        \[
            -\pdDif{\Gamma^{k}_{ii}}{x^j} - 2\pdDif{\Gamma^{k}_{ij}}{x^i} + 2\pdDif{\Gamma^{k}_{ij}}{x^j} + \pdDif{\Gamma^{k}_{jj}}{x^i} = 0,
        \]
        所以 
        \[
            \begin{aligned}
                0 ={} & \pdDif{\Gamma^{k}_{ii}}{x^j} + 2\pdDif{\Gamma^{k}_{ij}}{x^i} \\
                ={} & 3\pdDif{\Gamma^{k}_{ij}}{x^i} + \R^k_{iji} \\
                ={} & 3\pdDif{\Gamma^{k}_{ii}}{x^j} + 2\R^k_{iij}
            \end{aligned}
        \]
        因此 
        \[
            \pdDif{\Gamma^{k}_{ij}}{x^i} = -\dfrac{1}{3}\R^k_{iji} = -\dfrac{1}{3}(\R^k_{iji} + \R^{jii}),
        \]
        \[
            \pdDif{\Gamma^{k}_{ii}}{x^j} = -\dfrac{2}{3}\R^k_{iij}.
        \]

        取 $\mathbf{u} = u^i e_i + u^j e_j + u^k e_k$, 由上面的计算, 如果求和中 $i, j, k$ 只选到一个或两个下标, 那么这部分求和项为 $0$. 所以
        \[
            \begin{aligned}
                0 ={} & \pdDif{\Gamma^{k}_{ij}}{x^l} + \pdDif{\Gamma^{k}_{jl}}{x^i} + \pdDif{\Gamma^{k}_{il}}{x^j} \\
                ={} & 3\pdDif{\Gamma^{k}_{ij}}{x^l} + \bigl( \pdDif{\Gamma^{k}_{jl}}{x^i} - \pdDif{\Gamma^{k}_{ij}}{x^l} \bigr) + \bigl( \pdDif{\Gamma^{k}_{il}}{x^j} - \pdDif{\Gamma^{k}_{ij}}{x^l} \bigr) \\
                ={} & 3\pdDif{\Gamma^{k}_{ij}}{x^l} + \R^{k}_{jil} + \R^{k}_{ijl}.
            \end{aligned}
        \]
        因此
        \[
            \pdDif{\Gamma^{k}_{ij}}{x^l} = -\dfrac{1}{3}(\R^{k}_{jil} + \R^{k}_{ijl}).
        \]
        \item $Y(s)$ 满足平行移动方程
        \[
            \pdDif{\zeta^i(s)}{s} + \Gamma^{i}_{jk} \zeta^j(s) \xi^k(s) = 0,
        \]
        所以在 $q$ 点
        \[
            \pdDif{\zeta^i(s)}{s} = -\Gamma^{i}_{jk} \eta^j \xi^k = 0,
        \]
        并且利用 $\Gamma^i_{jk}(q) = 0$, 在 $q$ 点二阶导数为
        \[
            \pdDif{^2 \zeta^i(s)}{s^2} = -\pdDif{\Gamma^{i}_{jk}}{s}(q) \eta^j \xi^k.
        \]
        利用 \ref{enum::1} 的结论,
        \[
            \pdDif{\Gamma^{k}_{ij}}{s}(q) = -\dfrac{1}{3}(\R^{k}_{jil} + \R^{k}_{ijl})\xi^l,
        \]
        所以
        \[
            \begin{aligned}
                \zeta^i(s) ={} & \eta^i + \pdDif{\zeta^i(s)}{s} s + \dfrac{1}{2}\pdDif{^2 \zeta^i(s)}{s^2} s^2 + O(s^3) \\
                ={} & \eta^i + \dfrac{1}{6}(\R^i_{jkl} + \R^i_{kjl}) \eta^j \xi^k \xi^l s^2 + O(s^3).
            \end{aligned}
        \]
        因为 $\R^i_{jkl}$ 交换 $kl$ 会变号, 所以 $\R^i_{jkl}\eta^j \xi^k \xi^l = 0$. 因此
        \[
            \zeta^i(s) = \eta^i + \dfrac{1}{6}\R^i_{kjl} \xi^j \eta^k \xi^l s^2 + O(s^3).
        \]
        \item $\Inner{X_0}{Y_0} = 0$, 则 $\sum_{i} \eta^i \xi^i = 0$.
        所以
        \[
            \begin{aligned}
                ||Y(s)||^2_q ={} & g_{ij}(q)\xi^i(s) \xi^j(s) \\
                ={} & |\eta^i|^2 + \sum_{i}\dfrac{1}{6}\R_{ijkl} \eta^i \xi^j \eta^k \xi^l s^2 + O(s^3) \\
                ={} & 1 + \dfrac{1}{6}R(X_0, Y_0, X_0, Y_0)s^2 + O(s^3).
            \end{aligned}
        \]
    \end{enumerate}
\end{solution}

\subsection{179页 习题9}
\begin{problem}
    设 $m$ 维 Riemann 流形 $(M, g)$ 在测地极坐标系 $(r, \theta^1, \cdots, \theta^{m - 1})$ 下具有度量形式
    \[
        \dif s^2 = (\dif r)^2 + (f(r))^2 h_{ij}(\theta) \dif \theta^i \theta^j,
    \]
    其中 $m - 1$ 维度量 $\dif \sigma^2 = h_{ij}(\theta)\dif \theta^i \theta^j$ 具有常数截面曲率 $1$. 求证 $\dif s^2$ 具有常数截面曲率 $c$ 的充要条件是
    \[
        f(r) = \left\{
            \begin{array}{ll}
                \sin \bigl( \sqrt{c r^2} \bigr) / \sqrt{c} & c > 0, \\
                r & c = 0, \\
                \sinh \bigl( \sqrt{-c r^2} \bigr) / \sqrt{-c} & c < 0.
            \end{array}
        \right.
    \]
\end{problem}

\begin{solution}
    第一步要计算 Christoffel 记号.,
    \[
        \Gamma^{i}_{jk} = \dfrac{1}{2}g^{il}\left(\pdDif{g_{jl}}{x^k} + \pdDif{g_{lk}}{x^j} - \pdDif{g_{jk}}{x^l}\right).
    \]
    \begin{enumerate}
        \item 若三个指标都是 $r$:
        \[
            \Gamma^{r}_{rr} = 0.
        \]
        \item 若两个指标是 $r$, 一个是 $\theta^i$:
        \[
            \begin{aligned}
                \Gamma^{r}_{r i} ={} & \dfrac{1}{2}g^{rr}\left(\pdDif{g_{rr}}{\theta^i} + \pdDif{g_{ri}}{r} - \pdDif{g_{ri}}{r}\right) = 0,
            \end{aligned}
        \]
        \[
            \begin{aligned}
                \Gamma^{i}_{rr} ={} & \dfrac{1}{2}g^{ij}\left(\pdDif{g_{rj}}{r} + \pdDif{g_{jr}}{r} - \pdDif{g_{rr}}{\theta^j}\right) = 0.
            \end{aligned}
        \]
        \item 若一个指标是 $r$, 其余两个是 $\theta^i, \theta^j$:
        \[
            \begin{aligned}
                \Gamma^{r}_{ij} ={} & \dfrac{1}{2}g^{rr}\left(\pdDif{g_{ir}}{\theta^j} + \pdDif{g_{rj}}{\theta^i} - \pdDif{g_{ij}}{r}\right) = -f(r) f^{\prime}(r) h_{ij}(\theta), \\
                \Gamma^{i}_{rj} ={} & \dfrac{1}{2}g^{ik}\left(\pdDif{g_{rk}}{\theta^j} + \pdDif{g_{jk}}{r} - \pdDif{g_{rj}}{\theta^k}\right) = \dfrac{f^{\prime}(r)}{f(r)} \delta^i_{j}.
            \end{aligned}
        \]
        \item 若所有指标都不为 $r$: 设 $\bar{\Gamma}$ 表示 $\dif \sigma^2$ 的 Christoffel 记号,
        \[
            \Gamma^{i}_{jk} = \bar{\Gamma}^{i}_{jk}.
        \]
    \end{enumerate}
    那么 Riemann 曲率张量
    \[
        \R^{\alpha}_{\beta \delta \eta} = \pdDif{\Gamma^{\alpha}_{\beta \eta}}{x^\delta } + \Gamma^{\alpha}_{\delta \xi}\Gamma^{\xi}_{\beta \eta} - \pdDif{\Gamma^{\alpha}_{\beta \delta }}{x^\eta} - \Gamma^{\alpha}_{\eta \xi}\Gamma^{\xi}_{\beta \delta},
    \]
    所以,
    \[
        \begin{aligned}
            \R_{riri} ={} & g_{rr}\R^{r}_{iri} = \R^{r}_{i r i} \\
            ={} & \pdDif{\Gamma^{r}_{i i}}{r} + \Gamma^{r}_{r \xi}\Gamma^{\xi}_{i i} - \pdDif{\Gamma^{r}_{i r }}{\theta^i} - \Gamma^{r}_{i \xi}\Gamma^{\xi}_{i r} \\
            ={} & -(f(r))^2 h_{ii}(\theta) - f(r)f^{\prime\prime}(r) h_{ii}(\theta) - \sum_{j}\left(-f(r)f^\prime h_{ij}(\theta) \cdot \dfrac{f^\prime(r)}{f(r)}\delta^i_j\right) \\
            ={} & -f(r)f^{\prime\prime}(r) h_{ii}(\theta). \\
            \R_{ijij} = {} & g_{ik}\R^{k}_{jij} \\
            ={} & g_{ik}\left(\pdDif{\Gamma^{k}_{j j}}{\theta^i } + \Gamma^{k}_{i \xi}\Gamma^{\xi}_{j j} - \pdDif{\Gamma^{k}_{j i }}{\theta^j} - \Gamma^{k}_{j \xi}\Gamma^{\xi}_{j i}\right) \\
            ={} & g_{ik}\left(\pdDif{\Gamma^{k}_{j j}}{\theta^i } + \Gamma^{k}_{i l}\Gamma^{l}_{j j} - \pdDif{\Gamma^{k}_{j i }}{\theta^j} - \Gamma^{k}_{j l}\Gamma^{l}_{j i}\right) + g_{ik}\left( \Gamma^{k}_{i r}\Gamma^{r}_{j j} - \Gamma^{k}_{j r}\Gamma^{r}_{j i}\right)
        \end{aligned}
    \]
    设 $\dif \sigma^2$ 的曲率张量为 $\bar{R}$, 则有
    \[
        \begin{aligned}
            \bar{\R}^{i}_{j i j} ={} & \pdDif{\Gamma^{i}_{j j}}{\theta^i } + \Gamma^{i}_{i k}\Gamma^{k}_{j j} - \pdDif{\Gamma^{i}_{j i }}{\theta^j} - \Gamma^{i}_{j k}\Gamma^{k}_{j i}
        \end{aligned}
    \]
    所以
    \[
        \begin{aligned}
            \R_{ijij} ={} & (f(r))^2\bar{\R}_{ijij} + g_{ik}\left( \dfrac{f^{\prime}(r)}{f(r)} \delta^k_{i} \cdot -f(r) f^{\prime}(r) h_{jj}(\theta) - \dfrac{f^{\prime}(r)}{f(r)} \delta^k_{j} \cdot -f(r) f^{\prime}(r) h_{ij}(\theta) \right) \\
            ={} & (f(r))^2\bar{\R}_{ijij} - (f(r))^2 f^{\prime}(r)^2 \bigl(h_{ii}(\theta) h_{jj}(\theta) - h_{ij}(\theta)^2\bigr).
        \end{aligned}
    \]
    因此截面曲率为
    \begin{equation}\label{eq::sectional-curvature}
        \begin{aligned}
        \K(\partial r, \partial \theta^i) ={} & \dfrac{\R_{riri}}{g_{rr} g_{ii} - g_{ri}^2} = -\dfrac{f^{\prime\prime}(r)}{f(r)}, \\
        \K(\partial \theta^i, \partial \theta^j) ={} & \dfrac{\R_{ijij}}{g_{ii} g_{jj} - g{ij}^2} = \dfrac{1}{(f(r))^2} - \dfrac{(f(r))^2 (f^{\prime}(r))^2 \bigl(h_{ii}(\theta) h_{jj}(\theta) - h_{ij}(\theta)^2\bigr)}{f(r)^4 \bigl(h_{ii}(\theta) h_{jj}(\theta) - h_{ij}(\theta)^2\bigr)} \\
        ={} & \dfrac{1}{(f(r))^2} - \dfrac{(f^{\prime}(r))^2}{(f(r))^2}.
        \end{aligned}
    \end{equation}
    因此如果是常截面曲率流形, 那么
    \[
        -\dfrac{f^{\prime\prime}(r)}{f(r)} = \dfrac{1}{(f(r))^2} - \dfrac{(f^{\prime}(r))^2}{(f(r))^2} = c.
    \]
    即
    \[
        f^{\prime\prime}(r) = -cf(r), \qquad 1 - (f^{\prime}(r))^2 = c(f(r))^2.
    \]
    所以
    \begin{enumerate}
        \item 如果 $c > 0$, 那么 $f(r) = A \sin (\sqrt{c}r) + B \cos (\sqrt{c}r)$, 利用第二个方程,
        \[
            1 = c(f(r))^2 + (f^\prime (r))^2 = c B^2 + c A^2,
        \]
        此外, 计算 $|\partial B(r)|$ 的面积为
        \[
            A(r) = \int_{\partial B(r)} (f(r))^{2(m - 1)} \det (h_{ij}(\theta)) \dif \theta,
        \]
        当 $r \rightarrow 0$ 时, $A(r)$ 应当趋向于 $0$. 所以 $\lim_{r \rightarrow 0} f(r) = 0$. 所以 $A = \dfrac{1}{\sqrt{c}}$, $B = 0$. 即
        \[
            f(r) = \sin \bigl( r\sqrt{c} \bigr) / \sqrt{c}.
        \]
        \item 如果 $c = 0$, 那么 $f(r) = Ax + B$. 结合上面所说, 有
        \[
            1 = c(Ax + B)^2 + A^2, \qquad \lim_{r \rightarrow 0} f(r) = 0,
        \]
        所以 $A = 1$, $B = 0$. 即
        \[
            f(r) = r.
        \] 
        \item 如果 $c < 0$, 那么 $f(r) = A e^{r \sqrt{-c}} + B e^{-r \sqrt{-c}}$. 所以有
        \[
            1 = c(f(r))^2 + (f^\prime (r))^2 = 4c AB, \qquad \lim_{r \rightarrow 0} f(r) = A + B = 0.
        \]
        所以
        \[
            f(r) = e^{r \sqrt{-c}} / (2 \sqrt{-c}) + e^{-r \sqrt{-c}} / (2 \sqrt{-c}) = \sinh \bigl( r \sqrt{-c} \bigr)/ \sqrt{-c}.
        \]
    \end{enumerate}
    综上, 如果 $M$ 是常截面曲率流形, 且截面曲率为 $c$, 则 
    \[
        f(r) = \left\{
            \begin{array}{ll}
                \sin \bigl( r\sqrt{c} \bigr) / \sqrt{c} & c > 0, \\
                r & c = 0, \\
                \sinh \bigl( r \sqrt{-c} \bigr)/ \sqrt{-c} & c < 0.
            \end{array}
        \right.
    \]
    反过来, 如果 $f$ 表现如上, 那么根据 (\ref{eq::sectional-curvature}) 式, 可以计算截面曲率为 $c$. 因此题目得证.
\end{solution}

\subsection{问题 1.2}

\begin{problem}
    假设 $(\mfd, g)$ 是 $m$ 维 Riemann 流形, $p \in \mfd$ 是流形上任意一点. 在 $p$ 的法坐标 $(\mathcal{U}_p , x_i )$ 中证明:
    \begin{enumerate}[label = (\roman*)]
        \item $g_{ij} (x) = \delta_{ij} - \frac{1}{3} \R_{ikjl}(0) x_k x_l + O(|x|^3)$;
        \item $\det (g_{ij}) = 1 - \frac{1}{3} \Ric_{kl}(0) x_k x_l + O(|x|^3 )$.
    \end{enumerate}
\end{problem}

\begin{solution}
    \begin{enumerate}
        \item\label{enum::2} 核心是计算 $g_{ij}$ 沿由 $p$ 出发到 $x$ 的测地线的导数. 设 $x = (x^1, \cdots, x^n)$, 同时其也 $p$ 到 $x$ 指数映射的切向量. 设 $\gamma(t) = \exp_p(tx)$, 那么 $x = \gamma(1)$. 所以一阶导数为
        \[
            \dDif{}{t}g_{ij}(\gamma(t)) = x^k \partial_k\bigl( g_{ij}(\gamma(t)) \bigr).
        \]
        而 $\partial_k\left(g_{ij}\right) = \Inner{\Div{\partial_k}\partial_i}{\partial_j} + \Inner{\partial_i}{\Div{\partial_k}\partial_j} = g_{jl}\Gamma^{l}_{ik} + g_{il}\Gamma^{l}_{jk}$. 因为在法坐标的原点, Christoffel 记号为 $0$, 所以
        \[
            \dDif{}{t}g_{ij}(\gamma(t)) = 0.
        \]
        
        二阶导数为
        \[
            \begin{aligned}
            \dDif{^2}{t^2}g_{ij}(\gamma(t)) ={} & \dDif{}{t} \left[x^k \bigl( g_{jl}\Gamma^{l}_{ik} + g_{il}\Gamma^{l}_{jk} \bigr) \right] \\
            ={} & x^k\left( g_{jl}\dDif{\Gamma^l_{ik}}{t} + g_{il}\dDif{\Gamma^l_{jk}}{t} \right) \\
            ={} & x^k\left( g_{jl}x^m\pdDif{\Gamma^l_{ik}}{x^m} + g_{il}x^m\pdDif{\Gamma^l_{jk}}{x^m} \right).
            \end{aligned}
        \]
        在 $0$ 点, 利用第 \ref{sec::1} 节的 (\ref{enum::1}), 遂有
        \[
            \begin{aligned}
                \dDif{^2}{t^2}g_{ij}(0) ={} & x^k\left(x^m g_{jl} \pdDif{\Gamma^l_{ik}}{x^m}(0) + x^m g_{il} \pdDif{\Gamma^i_{jk}}{x^m}(0)\right) \\
                ={} & -\dfrac{1}{3}x^k x^m g_{jl}(0)(\R^l_{i k m}(0) + \R^{l}_{k i m}(0)) - \dfrac{1}{3}x^k x^m g_{il}(0)(\R^l_{j k m}(0) + \R^{l}_{k j m}(0)) \\
                ={} & -\dfrac{1}{3}x^k x^m (\R_{jikm}(0) + \R_{jkim}(0) + \R_{ijkm}(0) + \R_{ikjm}(0)) \\
                ={} & -\dfrac{2}{3}x^k x^l \R_{ijkl}(0),
            \end{aligned}
        \]
        代入 Taylor 展式, 得到
        \[
            g_{ij}(x) = \delta_{ij} - \dfrac{1}{3}\R_{ijkl}(0)x^k x^l + O(|x|^3).
        \]
        \item 依然要计算导数. 记 $G = (g_{ij})_{n \times n}$,
        \[
            \dDif{}{t}\det (g_{ij})(\gamma(t)) = \det G \tr G^{-1}G^{\prime} = \det (g_{ij})(\gamma(t))(g^{ki}g_{ik}^{\prime})(\gamma(t)),
        \]
        所以 $\dDif{}{t}\det (g_{ij})(0) = 0$. 继续计算, 二阶导数为
        \[
            \dDif{^2}{t^2}\det (g_{ij})(\gamma(t)) = \dDif{}{t}\sum_{j, k} \bigl[g^{jk} (g^{ki}g_{ij}^{\prime})\bigr](\gamma(t)),
        \]
        因为 $0 = (g^{ij}g_{jk})^{\prime} = (g^{ij})^{\prime}g_{jk} + g^{ij}(g_{jk})^{\prime}$, 在 $0$ 点处, $0 = (g^{ij})^{\prime}g_{jk}$. 所以 $(g^{ij})^{\prime}(0) = 0$. 因此上式中求导只会求在 $g^{\prime}$ 上. 于是利用 (\ref{enum::2}) 的结果,
        \[
            \dDif{^2}{t^2}\det (g_{ij})(0) = -\dfrac{2}{3}\sum_{j, k} \bigl[g^{jk} (g^{ki}x^k x^l \R_{ijkl}(0))\bigr] = -\dfrac{2}{3}x^k x^l \Ric_{kl}(0),
        \]
        代入 Taylor 展式, 得到
        \[
            \det (g_{ij})(x) = 1 - \dfrac{1}{3}\Ric_{kl}(0)x^k x^l + O(|x|^3).
        \]
    \end{enumerate}
\end{solution}

\newpage
\subsection{问题 1.3}

\begin{problem}
    假设 $(\mfd, g)$ 是 $m$ 维 Riemann 流形. 假设 $r > 0$ 足够小使得 $\exp_p \colon D_r (0) \rightarrow B_r (p)$ 是微分同胚, 其中 $D_r (0) = {v \in T_p \mfd : |v| < \varepsilon}$, $B_r (p) = \exp_p D_r (0)$. 记 $S_r (p) = \partial B_r (p)$. 证明:
    \begin{enumerate}
        \item $\vol (B_r (p)) = \omega_m r^m \left(1 - \dfrac{\scal(p)}{6(m + 2)}r^2 + O(r^3)\right)$, 其中 $\omega_m$ 是 $\real^m$ 的单位球的欧氏体积; 
        \item $\area (S_r (p)) = m \omega_m r^{m - 1} - \dfrac{1}{6} \scal(p) \omega_m r^{m + 1} + O(r^{m + 2})$.
    \end{enumerate}
\end{problem}

\begin{solution}
    \begin{enumerate}
        \item $\vol (B_r (p))$ 相当于在欧氏空间 $\bar{B}_r$ 上对 $\mfd$ 的体积元 $\dif \vol$ 积分. 而 $\dif \vol = \sqrt{\det (g_{ij})} \dif m_{\real^m}$. 设 
        \[
            \sqrt{\det (g_{ij})}(x) = 1 + a_k x^k + b_{kl}x^k x^l + O(|x|^3),
        \]
        那么平方可得
        \[
            a_k = 0, \qquad 2b_{kl} = - \dfrac{1}{3}\Ric_{kl}(p),
        \]
        所以
        \[
            \sqrt{\det (g_{ij})}(x) = 1 - \dfrac{1}{6}\Ric_{kl}(p)x^k x^l + O(|x|^3).
        \]
        
        因此积分得
        \[
            \begin{aligned}
                \vol (B_r (p)) ={} & \int_{\bar{B}_r} 1 - \dfrac{1}{6}\Ric_{kl}(p)x^k x^l + O(|x|^3) \dif x \\
                ={} & \vol(\bar{B}_r) - \dfrac{1}{6}\int_{\bar{B}_r}\Ric_{kl}(p)x^k x^l \dif x + \vol(\bar{B}_r)O(r^3) \\
            \end{aligned}
        \]
        考虑 $p$ 点附近的测地极坐标 $(r, \theta_1, \cdots, \theta_{m - 1})$. 则
        \[
            \left\{
            \begin{aligned}
                x_1 ={} & r \cos \theta_1 \\
                x_2 ={} & r \sin \theta_1 \cos \theta_2 \\
                \vdots{} & \\
                x_{m - 1} ={} & r \sin \theta_1 \cdots \sin \theta_{m - 2} \cos \theta_{m - 1} \\
                x_m ={} & r \sin \theta_1 \cdots \sin \theta_{m - 2} \sin \theta_{m - 1} 
            \end{aligned}
            \right.
        \]
        其中 $1 \leqslant i \leqslant m - 2$ 时, $\theta_i \in [0, \pi]$; 而 $\theta_{m - 1} \in [0, 2\pi]$. 观察对 $\theta_{m - 1}$ 的积分, 有以下结论:
        \begin{lemma*}
            若 $k \not= l$, 则
            \[
                \int_{\bar{B}_r}x^k x^l \dif x = 0.
            \]
        \end{lemma*}
        \begin{proof}[引理的证明]
            由于 $k \not= l$, 所以 $\theta_{m - 1}$ 的项只可能有三种情况: $\cos \theta_{m - 1}$, $\sin \theta_{m - 1}$, $\sin \theta_{m - 1}\cos \theta_{m - 1} = \sin (2\theta_{m - 1}) / 2$. 而这三者在 $[0, 2\pi]$ 上的积分都是 $0$. 因此引理成立.
        \end{proof}
        有此结论, 则 
        \[
            \int_{\bar{B}_r}\Ric_{kl}(p)x^k x^l \dif x = \int_{\bar{B}_r}\sum_{k}\Ric_{kk}(p)(x^k)^2 \dif x.
        \]
        由对称性, 不同 $k$ 的 $\int_{\bar{B}_r}(x^k)^2 \dif x$ 都相等, 并且其均等于
        \[
            \dfrac{1}{m}\int_{\bar{B}_r} \sum_{k = 1}^{m}(x^k)^2 \dif x = \dfrac{1}{m}\int_{\bar{B}_r} |x|^2 \dif x = \omega_m \dfrac{r^{m + 2}}{m + 2},
        \]
        故
        \[
            \int_{\bar{B}_r}\Ric_{kl}(p)x^k x^l \dif x = m\omega_m \dfrac{\scal(p)}{m + 2}r^{m + 2}.
        \]

        因此
        \[
            \vol (B_r (p)) = \omega_m r^m - \dfrac{1}{6}m\omega_m \dfrac{\scal(p)}{m + 2}r^{m + 2} + \omega_m r^mO(r^3) = \omega_m r^m(1 - \dfrac{\scal(p)}{6(m + 2)}r^2 + O(r^3)).
        \]
        \item 计算 $\area(S_r(p))$, 只需要对 $\vol(B_r(p))$ 关于 $r$ 求导即可.
    \end{enumerate}
\end{solution}

\subsection{问题 1.4}
\begin{problem}
    假设 $(\mfd, g)$ 是 $m$ 维 Riemann 流形, $p \in M$, $\Pi_p \subset T_p \mfd$ 是 2 维截面. 令 $C_r^0 = \{v \in \Pi_p : |v| = r\}$, $C_r = \exp_p C_r^0$. 用 $L_r$ 记曲线 $C_r$ 的长度. 证明:
    \[
        \K(\Pi_p) = \dfrac{3}{\pi}\lim_{r \rightarrow 0}\dfrac{2\pi r - L_r}{r^3},
    \]
    其中 $\K(\Pi_p)$ 表示 $p$ 点处关于 $\Pi_p$ 的截面曲率.
\end{problem}

\begin{solution}
    对 $\exp_p \Pi_p$ 应用问题 1.3 的结论. 这是一个 $2$ 维的子流形, 其截面曲率就是 $K(\Pi_p)$. 因此
    \[
        L_r = \area(C_r) = 2\pi r - \dfrac{2K(\Pi_p)}{6}\pi r^3 + O(r^4).
    \]

    故
    \[
        \lim_{r \rightarrow 0}\dfrac{2\pi r - L_r}{r^3}
        = \lim_{r \rightarrow 0} \dfrac{\dfrac{2K(\Pi_p)}{6}\pi r^3 + O(r^4)}{r^3} = \dfrac{\pi}{3}K(\Pi_p),
    \]
    问题得证.
\end{solution}

\subsection{问题 1.5}

\begin{problem}
    假设 $(\mfd, g)$ 是 $m$ 维 Riemann 流形, $\gamma_k \colon [0, 1] \rightarrow \mfd$, $k = 1, 2$ 是两条正规测地线 (速度为 $1$ 的测地线), 且满足 $\gamma_1 (0) = \gamma_2 (0) = p$. 用 $\theta \in (0, \pi)$ 记 $v = \dot{\gamma}_1 (0)$ 与 $w = \dot{\gamma}_2(0)$ 的夹角, $\Pi_p = \mathrm{span}\{v, w\}$ 是 $T_p \mfd$ 的 $2$ 维截面. 证明:
    \[
        d(\gamma_1(t), \gamma_2(t)) = \sqrt{2(1 - \cos \theta)} \bigl[ 1 - \dfrac{1}{6}\K(\Pi_p) \cos^2(\theta / 2) t^2 + O(t^3) \bigr]t.
    \]
\end{problem}

\begin{solution}
    为计算距离, 设 $A(t)$ 为 $\gamma_2(t) = tw$ 关于 $\gamma_1(t) = tv$ 指数映射的切向量, 即 $\exp_{tv}A(t) = tw$. 考虑 $\sigma(t, s) := \exp_{tv}sA(t)$. $T := \sigma_{\ast}\partial_s$ 是 $tv$ 到 $tw$ 测地线 $a_t \colon s \mapsto \sigma(t, s)$ 的切向量场, 且测地线 $a_t$ 的长度为 
    \[
        L(a_t) = |T(t, 0)| = d(\gamma_1(t), \gamma_2(t)),
    \]
    而 $J := \sigma_{\ast}\partial_t$ 是 Jacobi 场, 满足
    \[
        \ddot{J} = \Rm(T, J)T,
    \]
    借此, 计算 $L^2 := (L(a_t))^2 = |T(t, 0)|^2$ 在 $p$ 点关于 $t$ 的导数:
    \begin{itemize}
        \item $1$ 阶:
        \[
            \dDif{L^2}{t}(p) = \Div{J}\Inner{T}{T}(p) = 2\Inner{\Div{J}T}{T}(0, 0) = 0.
        \]
        \item $2$ 阶: 因为 $J(t, 0) = v$, $J(t, 1) = w$, $p$ 处 $\ddot{J} = 0$, 所以 $J(0, s) = v + s(w - v)$. 即 $\Div{J}T(0, s) = \Div{T}J(0, s) = w - v$. 利用此结果, 以及 $T(0, s) = 0$,
        \[
            \begin{aligned}
                \dDif{^2 L^2}{t^2}(0, 0) ={} & 2\Inner{\Div{J}\Div{J}T}{T}(p) + 2\Inner{\Div{J}T}{\Div{J}T}(0, 0) \\
                ={} & 2|w - v|^2 = 4(1 - \cos \theta).
            \end{aligned}
        \]
        \item $3$ 阶: 利用 $T(0, s) = 0$, 以及 $2\Inner{\Div{J}\Div{J}T}{\Div{J}T}(0, s) = J(\Inner{\Div{J}T}{\Div{J}T})(0, s) = J(2|v - w|^2) = 0$,
        \[
            % \begin{aligned}
                \dDif{^3 L^2}{t^3}(0, 0) = 2\Inner{\Div{J}\Div{J}\Div{J}T}{T}(0, 0) + 2\Inner{\Div{J}\Div{J}T}{\Div{J}T}(0, 0) + 4\Inner{\Div{J}\Div{J}T}{\Div{J}T}(0, 0) = 0
            % \end{aligned}
        \]
        \item $4$ 阶: 作为铺垫, 首先计算
        \[
            \begin{aligned}
                \Div{J}\Div{J}T ={} & \Div{J}\Div{T}J \\
                ={} & \Rm(J, T)J - \Div{T}\Div{J}J,
            \end{aligned}
        \]
        和
        \[
            \begin{aligned}
                {} & \Div{J}\Div{J}\Div{J}T(0, 0) \\
                ={} & \Div{J}(\Rm(J, T)J) - \Div{J}\Div{T}\Div{J}J \\
                ={} & (\Div{J}\Rm)(J, T)J + \Rm(\Div{J}J, T)J + \Rm(J, \Div{J}T)J + \Rm(J, T)\Div{J}J - \Div{J}\Div{T}\Div{J}J \\
                ={} & \Rm(J, \Div{J}T)J(0, 0) - \Div{J}\Div{T}\Div{J}J(0, 0),
            \end{aligned}
        \]
        其中, 最后一个等式用到了 $T(0, s) = 0$, 所以只要张量中有单独出现的 $T$, 就必须为 $0$. 此外, 借助这一点, 可以继续计算得到 $\Div{T}\Div{J}J(0, s) = \Div{J}\Div{T}J(0, s) = \Div{J}(w - v) = 0$; $\Div{J}J(0, 0) = 0$, $\Div{J}J(0, 1) = 0$. 其关于 $s$ 的导数 $\Div{T}\Div{J}J(0, s)$ 为 $0$, 所以 $\Div{J}J(0, s) = 0$. 利用以上结果,
        \[
            \begin{aligned}
                {} & \Inner{\Div{J}\Div{T}\Div{J}J}{\Div{J}T}(0, 0) \\
                ={} & J\Inner{\Div{T}\Div{J}J}{\Div{J}T}(0, 0) - \Inner{\Div{T}\Div{J}J}{\Div{J}\Div{J}T}(0, 0) \\
                ={} & JT\Inner{\Div{J}J}{\Div{J}T}(0, 0) - J\Inner{\Div{T}\Div{J}T}{\Div{J}J}(0, 0) \\
                ={} & TJ\Inner{\Div{J}J}{\Div{J}T}(0, 0) - \Inner{\Div{J}\Div{T}\Div{J}T}{\Div{J}J}(0, 0) - \Inner{\Div{T}\Div{J}T}{\Div{J}\Div{J}J}(0, 0) \\
                ={} & TJ\Inner{\Div{J}J}{\Div{J}T}(0, 0).
            \end{aligned}
        \]
        所以可以计算 $4$ 阶导数
        \[
            \begin{aligned}
                \dDif{^4 L^2}{t^4}(0, 0) ={} & 8\Inner{\Div{J}\Div{J}\Div{J}T}{\Div{J}T}(0, 0) \\
                ={} & 8\Inner{\Rm(J, \Div{J}T)J}{\Div{J}T}(0, 0) - TJ\Inner{\Div{J}J}{\Div{J}T}(0, 0) \\
                ={} & -8\Rm(w - v, v, w - v, v) - TJ\Inner{\Div{J}J}{\Div{J}T}(0, 0).
            \end{aligned}
        \]
        因为上式中, 将 $(0, 0)$ 替换为 $(0, s)$ 也是成立的($L = |T(0, s)|$), 因此 $TJ\Inner{\Div{J}J}{\Div{J}T}(0, s)$ 是和 $s$ 无关的常数. 而 
        \[
            J\Inner{\Div{J}J}{\Div{J}T}(0, t) = \Inner{\Div{J}\Div{J}J}{\Div{J}T}(0, t) + \Inner{\Div{J}J}{\Div{J}\Div{J}T}(0, t),
        \]
        所以 $J\Inner{\Div{J}J}{\Div{J}T}(0, 0) = J\Inner{\Div{J}J}{\Div{J}T}(0, 1) = 0$. 记 $\alpha(s) := J\Inner{\Div{J}J}{\Div{J}T}(0, s)$, 所以 $\alpha^\prime(s) = \mathrm{const.}$, $\alpha(0) = \alpha(1) = 0$, 所以 $\alpha(s) \equiv 0$. 因此 $TJ\Inner{\Div{J}J}{\Div{J}T}(0, s) \equiv 0$. 所以
        \[
            \dDif{^4 L^2}{t^4}(0, 0) = -8\Rm(w - v, v, w - v, v) = -8\K(\Pi_p)(1 - \cos^2 \theta).
        \]
    \end{itemize}
    所以, 用 Taylor 展式
    \[
        L^2 = 2(1 - \cos \theta) t^2 - \dfrac{8}{4!}\K(\Pi_p)(1 - \cos^2 \theta) t^4 + O(t^5),
    \]
    因此用待定系数法可以算出
    \[
        L = \sqrt{2(1 - \cos \theta)} \bigl[t - \dfrac{1}{6}\K(\Pi_p) \cos^2(\theta / 2) t^3 + O(t^4) \bigr].
    \]
\end{solution}

\end{document}