\documentclass{ctexart}

\usepackage{amsmath}
\usepackage{amssymb}
\usepackage{amsthm}
\usepackage{enumitem}
\usepackage{hyperref}

\usepackage{mathrsfs} % for \VecFld
\usepackage{stmaryrd} % for \owedge

\usepackage{xcolor, framed}
\definecolor{shadecolor}{RGB}{241, 241, 255}
\newcounter{problemname}
\newenvironment{problem}{\begin{shaded}\stepcounter{problemname}\par\noindent\textbf{题目\arabic{problemname}. }}{\end{shaded}\par}
\newenvironment{solution}{\par\noindent\textbf{解答. }}{\par}

\newcommand{\cred}{\textcolor{red}}
\newcommand{\cblue}{\textcolor{blue}}
\newcommand{\ccyan}{\textcolor{cyan}}
\newcommand{\cmag}{\textcolor{magenta}}
\newcommand{\cpink}{\textcolor{pink}}

\usepackage{bbm}

\usepackage{geometry}
\geometry{a4paper,scale=0.8}

% number fields
\newcommand{\real}{\mathbb{R}}
\newcommand{\Zahl}{\mathbb{Z}}
\newcommand{\half}{\mathbb{H}}

% function spaces
\newcommand{\cont}[1]{C^{#1}}

% manifold spaces
\newcommand{\sphere}{\mathbb{S}}        % a big S

% unitary operators
\def\dif{\mathop{}\!\mathrm{d}}         % differential
\DeclareMathOperator{\tr}{tr}           % trace
\newcommand{\Inner}[2]{\langle {#1}, {#2} \rangle}            % inner product
\newcommand{\norm}[1]{\Vert {#1} \Vert}

% differential manifold
\newcommand{\VecFld}{\mathscr{X}}
\newcommand{\vol}{\mathrm{vol}}

% Riemannian connection
\newcommand{\Rm}{\mathrm{Rm}}           % curvature tensor
\newcommand{\R}{\mathrm{R}}             % curvature tensor, sometimes scalar curvature
\newcommand{\K}{\mathrm{K}}             % sectional curvature
\newcommand{\Ric}{\mathrm{Ric}}         % Ricci tensor
\newcommand{\scal}{\mathrm{scal}}       % scalar curvature

\newcommand{\Div}[1]{\nabla_{#1}}

% submanifold
\newcommand{\bnabla}{\overline{\nabla}}
\newcommand{\tnabla}{\widetilde{\nabla}}
\newcommand{\bRm}{\overline{\Rm}}
\newcommand{\tRm}{\widetilde{\Rm}}

\newcommand{\bDiv}[1]{\bnabla_{#1}}
\newcommand{\tDiv}[1]{\tnabla_{#1}}

\newcommand{\Shape}[1]{A_{#1}}
\newcommand{\Secd}{\mathbf{h}}
\newcommand{\Second}[2]{\mathbf{h}(#1, #2)}
\newcommand{\MeanCV}{\mathbf{H}}

\newcommand{\pdDif}[2]{\dfrac{\partial #1}{\partial #2}}

\newcommand{\dDif}[2]{\dfrac{\dif #1}{\dif #2}}



\title{第七次作业}
\author{洪艺中 12335025}
\begin{document}
\maketitle
\subsection{149 页 问题 15}
\begin{problem}
    设 $ \sphere^m = \bigl\{ x \in \real^{m + 1} : \sum_{i} (x^i)^2 = r^2, r > 0 \bigr\} $. 作球极投影
    \[
        \phi \colon \sphere^m(r) \setminus \{(0, \cdots, 0, r)\} \rightarrow \real^m.
    \]
    证明: $\phi$ 为共形映射, 即对于 Riemann 流形 $(\sphere^m(r), \tilde{g}$ 和 $(\real^m, g)$, 有 $\tilde{g} = \phi^{\ast}g$, 这里 $\tilde{g}$ 是 $\sphere^m(r) \subset \real^{m + 1}$ 的诱导度量, $g$ 为 $\real^m$ 上的欧氏度量.
\end{problem}
\begin{solution}
    设 $\phi$ 将 $\sphere^{m}$ 投影到 $\real^{m + 1}$ 的 $x^{n + 1} = 0$ 平面, 也就是 $\real^m$. 设 $\real^m$ 的坐标是 $(y_1, \cdots, y_m)$, 那么 $\phi$ 可以表达为
    \[
        \phi(y_1, \cdots, y_n) = \biggl( \dfrac{2r^2y_1}{|y|^2 + r^2}, \cdots, \dfrac{2r^2y_m}{|y|^2 + r^2}, r\dfrac{|y|^2 - r^2}{|y|^2 + r^2} \biggr),
    \]
    计算切映射,
    \newcommand{\rightpart}[1]{#1}
    \[
        \phi^{\ast} = \dfrac{2r^2}{|y|^2 + r^2}
        \begin{bmatrix}
            \mathrm{I}_{m} & 0
        \end{bmatrix}
        - \dfrac{4 r^2}{(|y|^2 + r^2)^2}
        \begin{bmatrix}
            \rightpart{y_1^2} & \rightpart{y_1y_2} & \cdots & \rightpart{y_1y_m} & -ry_1 \\
            \rightpart{y_2y_1} & \rightpart{y_2^2} & \cdots & \rightpart{y_2y_m} & -ry_2 \\
            \cdots & \cdots & \ddots & \cdots & \vdots \\
            \rightpart{y_my_1} & \rightpart{y_my_2} & \cdots & \rightpart{y_m^2} & -ry_m
        \end{bmatrix}.
    \]
    容易发现, 若 $i \not= j$, 那么 $\Inner{\partial_{y_i}}{\partial_{y_j}}_{\tilde{g}} = 0$, 这是因为内积为
    \[
        \begin{aligned}
             {} & \Inner{\partial_{y_i}}{\partial_{y_j}}_{\tilde{g}} \\
            ={} & -\dfrac{2r^2}{|y|^2 + r^2} \cdot \dfrac{4 r^2}{(|y|^2 + r^2)^2} (y_iy_j + y_jy_i) + \biggl(\dfrac{4 r^2}{(|y|^2 + r^2)^2} \biggr)^2 \sum_{k = 1}^{m} \left( y_k y_i \cdot y_k y_j \right) + r^2 y_i y_j \\
            ={} & \dfrac{2r^2}{|y|^2 + r^2} \cdot \dfrac{4 r^2}{(|y|^2 + r^2)^2}(-2y_i y_j + 2y_i y_j) \\
            ={} & 0.
        \end{aligned}
    \]
    因此 $\tilde{g}$ 是对角的. 
    \[
        \begin{aligned}
             {} & \Inner{\partial_{y_i}}{\partial_{y_i}}_{\tilde{g}} \\
            ={} & \biggl( \dfrac{2r^2}{|y|^2 + r^2} \biggr)^2 - \dfrac{2r^2}{|y|^2 + r^2} \cdot \dfrac{4 r^2}{(|y|^2 + r^2)^2} 2y_i^2
            + \biggl( \dfrac{4 r^2}{(|y|^2 + r^2)^2} \biggr)^2 (\sum_{k = 1}^{m} \left( (y_k)^2 (y_i)^2 \right) - r^2 (y_i)^2) \\
            ={} & \dfrac{2r^2}{|y|^2 + r^2} \cdot \dfrac{4 r^2}{(|y|^2 + r^2)^2} \biggl( \dfrac{1}{2}(|y|^2 + r^2) - 2(y^i)^2 + 2(y^i)^2 \biggr) \\
            ={} & \dfrac{4r^4}{(|y|^2 + r^2)^2}.
        \end{aligned}
    \]
    即 $\tilde{g} = \dfrac{4r^4}{(|y|^2 + r^2)^2}g$, 所以 $\tilde{g}$ 是共形变换.
\end{solution}

\subsection{问题 1.2}
\newcommand{\coeff}{\biggl( 1 + \dfrac{m}{2 \rho}\biggr)}
\begin{problem}
    在 $M = \real^{3} \setminus \{0\}$ 上配上度量 $g = \coeff^4\delta$, 其中 $m > 0$ 为一给定常数, $\rho(x) = \sqrt{(x_1)^2 + (x_2)^2 + (x_3)^2}$ 为点 $x = (x_1, x_2, x_3)$ 到原点的欧氏距离, $\delta$ 是三维欧氏空间 $\real^3$ 上的标准欧氏度量.
    \begin{enumerate}
        \item 计算 $(M, g)$ 的数量曲率;
        \item 令 $S_r = \{ x \in \real^3 : (x_1)^2 + (x_2)^2 + (x_3)^2 = r^2 \}$, $r > 0$. 计算 $S_r$ 在 Riemann 流形 $(M, g)$ 中的面积 $|S_r|$. 并指出 $r$ 取何值时, $|S_r|$ 最小, 最小值为何?
        \item 计算 $\frac{1}{16}  \int_{S_r} (\partial_j g_{ii} - \partial_i g_{ij})n^j \dif x$, 其中 $n^j$ 是 $S_r$ 内法向量的第 $j$ 个分量.
    \end{enumerate}
\end{problem}
\begin{solution}
    \begin{enumerate}
        \item 利用共形变换的公式, $ \phi = \coeff^2 $,
        \[
            \partial_i(\log \phi) = \dfrac{2}{1 + \dfrac{m}{2\rho}} \cdot -\dfrac{mx_i}{2\rho^3} = -\dfrac{mx_i}{\rho^2 (\rho + \dfrac{m}{2})},
        \]
        所以
        \[
            V = \nabla \log \phi = -\dfrac{m}{\rho^2 (\rho + \dfrac{m}{2})}\mathbf{x}, \quad \omega = -\dfrac{m}{\rho^2 (\rho + \dfrac{m}{2})}\dif (\dfrac{\rho^2}{2}).
        \]
        其中 $\mathbf{x}$ 是位置向量, $\omega(X) = \Inner{X}{V}_{g}$.

        故
        \[
            \psi(X, Y) = (\Div{X}\omega)(Y) - \omega(X)\omega(Y) + \dfrac{1}{2}\omega(V)\Inner{X}{Y}_{g},
        \]
        其迹为
        \[
            \tr \psi = \Delta \log \phi - \Inner{V}{V}^2_{g} + \Inner{V}{V}^2_{g} = -\dfrac{ -3\rho - \frac{5m}{2} }{ \rho^2 (\rho + \dfrac{ m }{ 2 })^2 }.
        \]
        所以
        \[
            \coeff^4 \scal_g = \scal_{\delta} - 2(3 - 1) \tr \psi,
        \]
        \[
            \scal_{g} = \dfrac{ (12 \rho + 10 m) \rho^2 }{ (\rho + \dfrac{ m }{ 2 })^6 }.
        \]
        \item 设 $\real^m$ 上的极坐标为 $(r, \theta_i)$, 对应的度量为 $\dif s^2 = \dif r^2 + k_{ij}\dif \theta_i \dif \theta_j$. 那么在 $S_r$ 上, 取 $\theta_i$ 构成的坐标, 那么
        \[
            (\dif s^2)_{S_r} = (1 + \frac{m}{2r})^4 k_{ij} \dif \theta_i \dif \theta_j.
        \]
        所以
        \[
            \vol(S_r)_{g} = (1 + \frac{m}{2r})^4 \vol(S_r)_{\delta} = 4 \pi r^2 (1 + \frac{m}{2r})^4 \geqslant 4\pi m^2 .
        \]
        其中, $r = \dfrac{m}{2}$ 时取等, 也就是取得最小值 $4\pi m^2 $.
        \item 设 $\mathbf{x} = (x^1, x^2, x^3) \in \real^3 \setminus \{0\}$ 处外法向量 $\mathbf{n} = n^j\partial_j$, 因为 $\mathbf{n}$ 和 $\mathbf{x}$ 平行, 所以可以设 $n^j = kx^j$. 则
        \[
            \sum_{j = 1}^{3} (n^j)^2\coeff^4 = k^2|\mathbf{x}|^2\coeff^4 = 1,
        \]
        所以
        \[
            k = -\dfrac{1}{\rho \coeff^2}.
        \]

        对题目中的式子, $j = i$ 时, $\partial_j g_{ii} = \partial_i g_{ij})$, 其余情况下, $g_{ij} = 0$, 所以
        \[
            (\partial_j g_{ii} - \partial_i g_{ij})n^j = \sum_{j \not= i} \partial_j g_{ii} n^j.
        \]
        因此
        \[
            (\partial_j g_{ii} - \partial_i g_{ij})n^j = 2 \coeff \dfrac{m}{\rho^4}(\rho^2 - (x^i)^2).
        \]
        在 $S_r$ 上, $\rho = r$, 代入积分得
        \[
            \begin{aligned}
            \frac{1}{16}\int_{S_r} (\partial_j g_{ii} - \partial_i g_{ij})n^j \dif x 
            ={} & \dfrac{1}{16}\dfrac{2m}{r^4}\biggl( 1 + \dfrac{m}{2 r}\biggr) \int_{S_r} (r^2 - (x^i) ^2) \dif x \\
            ={} & \dfrac{1}{16} \dfrac{2m}{r^4}\biggl( 1 + \dfrac{m}{2 r}\biggr)^5 \int_{-r}^{r}\int_{\{(x^j, x^k) : (x^j)^2 + (x^k)^2 = r^2 - (x^i)^2\}}(r^2 - (x^i) ^2) \dif x^j \dif x^k \dif x^i \\
            ={} & \dfrac{1}{16} \dfrac{2m}{r^4}\biggl( 1 + \dfrac{m}{2 r}\biggr)^5 \int_{-r}^{r} 2\pi\sqrt{r^2 - (x^i)^2}(r^2 - (x^i)^2) \dif x^i \\
            ={} & \dfrac{\pi^2}{32} \dfrac{2m}{r^4}\biggl( 1 + \dfrac{m}{2 r}\biggr)^5
            \end{aligned}
        \]
    \end{enumerate}
\end{solution}

\newpage
\subsection{问题 1.3}
\begin{problem}
    给定 $m$ 维 Riemann 流形 $(M, g)$.
    \begin{enumerate}
        \item 假设 $T$ 是 $M$ 上对称的 $(0, 2)$ 型张量, $\tnabla$ 是共形度量 $\tilde{g} = \phi^2 g$ (这里 $\phi$ 是 $M$ 上的光滑正函数) 的 Levi-Civita 联络. 证明: 任取 $X, Y, Z \in \VecFld(m)$,
        \[
            \tDiv{Z}T(X, Y) - \tDiv{Y}T(X, Z) - \Div{Z}T(X, Y) + \Div{Y}T(X, Z) = T \owedge g(V, X, Y, Z),
        \]
        其中 $V = \nabla \log \phi$, $\owedge$ 为 Kulkarni-Nomizu 乘积.
        \item 假设 $m \geqslant 3$, 证明: 任取 $X, Y, Z \in \VecFld(M)$,
        \[
            \tilde{C}(X, Y, Z) = C(X, Y, Z) + W(V, X, Z, Y),
        \]
        其中 $\tilde{C}$ 和 $C$ 分别是度量 $\tilde{g} = \phi^2g$ 和 $g$ 确定的 Cotton 张量, $W$ 是 Weyl 张量, $V = \nabla \log \phi$.
    \end{enumerate}
\end{problem}
\begin{solution}
    \begin{enumerate}
        \item\label{enum::1} 利用
        \[
            \tDiv{X}Y - \Div{Y}X = \omega(X)Y + \omega(Y)X - g(X, Y)V,
        \]
        其中 $\omega(v) = g(v, V), \forall v \in \VecFld(M)$. 得到
        \[
            \begin{aligned}
                 {} & \tDiv{Z}T(X, Y) - \Div{Z}T(X, Y) \\
                ={} & -T(\tDiv{Z}X - \Div{Z}X, Y) - T(X, \tDiv{Z}Y - \Div{Z}Y) \\
                ={} & -\omega(Z)T(X, Y) - \omega(X)T(Z, Y) + g(X, Z)T(V, Y) - \omega(Z)T(X, Y) - \omega(Y)T(X, Z) + g(Z, Y)T(X, V).
            \end{aligned}
        \]
        所以
        \[
            \begin{aligned}
                  & \tDiv{Z}T(X, Y) - \tDiv{Y}T(X, Z) - \Div{Z}T(X, Y) + \Div{Y}T(X, Z) \\
                ={} & \cblue{\omega(Y)T(X, Z)} + \cred{\omega(X)T(Y, Z)} - g(X, Y)T(V, Z) + \cblue{\omega(Y)T(X, Z)} + \ccyan{\omega(Z)T(X, Y)} - \cmag{g(Y, Z)T(X, V)} \\
                -{\ } & \ccyan{\omega(Z)T(X, Y)} - \cred{\omega(X)T(Y, Z)} + g(X, Z)T(V, Y) - \ccyan{\omega(Z)T(X, Y)} - \cblue{\omega(Y)T(X, Z)} + \cmag{g(Y, Z)T(X, V)} \\
                ={} & \omega(Y)T(X, Z) - g(X, Y)T(V, Z) - \omega(Z)T(X, Y) + g(X, Z)T(V, Y) \\
                ={} & T \owedge g (V, X, Y, Z).
           \end{aligned}
        \]
        \item 记 $A$ 为 Schouten 张量, 则
        \[
            W = \Rm - A \owedge g, \quad C(X, Y, Z) = \Div{Z}A(X, Y) - \Div{Y}A(X, Z),
        \]
        以及 % \phi(X, Y) = g(X, V)Y + g(Y, V)X - g(X, Y)V
        \[
            \tilde{A}(X, Y) = A(X, Y) - \psi(X, Y), \quad \psi(X, Y) = \nabla^2 \log \phi (X, Y) - \omega(X)\omega(Y) + \dfrac{1}{2}\omega(V)g(X, Y),
        \]
        利用 \ref{enum::1}, 
        \[
            \tilde{C}(X, Y, Z) - C(X, Y, Z) = A \owedge g(V, X, Y, Z) - \tDiv{Z}\psi(X, Y) + \tDiv{Y}\psi(X, Z). \tag{$\ast$}\label{eq::1}
        \]
        我没发现有什么特别好的计算方法, 只能展开了\ldots{}

        首先
        \[
            % model
            % coefg(\Div{second}V, first) - coefg(first, V)g(second, V) + \dfrac{1}{2}g(V, V)coefg(first, second)
            % negative model
            % -coefg(\Div{second}V, first) + coefg(first, V)g(second, V) - \dfrac{1}{2}g(V, V)coefg(first, second)
            \begin{aligned}
                 {  } & \tDiv{Z}\psi(X, Y) \\
                ={  } & Z(\psi(X, Y)) - \psi(\tDiv{Z}X, Y) - \psi(X, \tDiv{Z}Y) \\
                ={  } & Z\bigl(g(\Div{Y}V, X) - g(X, V)g(Y, V) + \dfrac{1}{2}g(V, V)g(X, Y)\bigr) \\
                -{\ } & g(\Div{Y}V, \Div{Z}X) + g(\Div{Z}X, V)g(Y, V) - \dfrac{1}{2}g(V, V)g(\Div{Z}{X}, Y) \\
                -{\ } & g(Z, V)g(\Div{Y}V, X) + \cred{g(X, V)g(Y, V)g(Z, V)} - \dfrac{1}{2}\cpink{g(V, V)g(X, Y)g(Z, V)} \\
                -{\ } & {g(X, V)g(\Div{Y}V, Z)} + \cred{g(X, V)g(Y, V)g(Z, V)} - \dfrac{1}{2}\cblue{g(V, V)g(X, V)g(Y, Z)} \\
                +{\ } & g(X, Z)g(\Div{Y}V, V) - \ccyan{g(V, V)g(X, Z)g(Y, V)} + \dfrac{1}{2}\ccyan{g(V, V)g(X, Z)g(Y, V)} \\
                -{\ } & g(\Div{\Div{Z}Y}V, X) + g(X, V)g(\Div{Z}Y, V) - \dfrac{1}{2}g(V, V)g(X, \Div{Z}Y) \\
                -{\ } & g(Z, V)g(\Div{Y}V, X) + \cred{g(X, V)g(Y, V)g(Z, V)} - \dfrac{1}{2}\cpink{g(V, V)g(X, Y)g(Z, V)} \\
                -{\ } & g(Y, V)g(\Div{Z}V, X) + \cred{g(X, V)g(Y, V)g(Z, V)} - \dfrac{1}{2}\ccyan{g(V, V)g(X, Z)g(Y, V)} \\
                +{\ } & g(Y, Z)g(\Div{X}V, V) - \cblue{g(V, V)g(X, V)g(Y, Z)} + \dfrac{1}{2}\cblue{g(V, V)g(X, V)g(Y, Z)},
            \end{aligned}
        \]
        由于接下来要计算的是 $-\tDiv{Z}\psi(X, Y) + \tDiv{Y}\psi(X, Z)$, 因此如果上面的某项关于 $Y$ 和 $Z$ 是对称的, 那么在作差中就会消去. \textbf{注意到}, 这样的项有\cred{红}、\cblue{蓝}、``\ccyan{青} $+$ \cpink{粉}''三种. 所以 在接下来的求和中, 可以忽略这些项.
        \[
            \begin{aligned}
                {  } & \tDiv{Z}\psi(X, Y) \\
               ={  } & g(\Div{Z}\Div{Y}V, X) + \cmag{g(\Div{Y}V, \Div{Z}X)} - \ccyan{g(\Div{Z}X, V)g(Y, V)} \\
               -{\ } & \textcolor{brown}{g(X, \Div{Z}V)g(Y, V)} - \cpink{g(X, V)g(\Div{Z}Y, V)} - \textcolor{purple}{g(X, V)g(Y, \Div{Z}V)} \\
               +{\ } & g(\Div{Z}V, V)g(X, Y) + \cblue{\dfrac{1}{2}g(V, V)g(\Div{Z}X, Y)} + \cred{\dfrac{1}{2}g(V, V)g(X, \Div{Z}Y)} \\
               -{\ } & \cmag{g(\Div{Y}V, \Div{Z}X)} + \ccyan{g(\Div{Z}X, V)g(Y, V)} - \cblue{\dfrac{1}{2}g(V, V)g(\Div{Z}X, Y)} \\
               -{\ } & g(\Div{\Div{Z}Y}V, X) + \cpink{g(X, V)g(\Div{Z}Y, V)} - \cred{\dfrac{1}{2}g(V, V)g(X, \Div{Z}Y)} \\
               +{\ } & g(X, Z)g(\Div{Y}V, V) + g(Y, Z)g(\Div{X}V, V) - 2g(Z, V)g(\Div{Y}V, X)  \\
               -{\ } & \textcolor{purple}{g(X, V)g(\Div{Y}V, Z)} - \textcolor{brown}{g(Y, V)g(\Div{Z}V, X)} \\
               ={  } & g(\Div{Z}\Div{Y}V, X) - 2\textcolor{brown}{g(Y, V)g(X, \Div{Z}V)} - \textcolor{purple}{2g(X, V)g(Y, \Div{Z}V)} \\
               +{\ } & g(\Div{Z}V, V)g(X, Y) - g(\Div{\Div{Z}Y}V, X) + g(X, Z)g(\Div{Y}V, V) \\
               +{\ } & g(Y, Z)g(\Div{X}V, V) - 2g(Z, V)g(\Div{Y}V, X) \\
               ={  } & g(\Div{Z}\Div{Y}V, X) - g(\Div{\Div{Z}Y}V, X) \\
               +{\ } & \bigl[ g(Y, Z)g(\Div{X}V, V) \bigr] + \bigl[ g(\Div{Z}V, V)g(X, Y) + g(X, Z)g(\Div{Y}V, V) \bigr] \\
               -{\ } & 2\bigl[ g(Z, V)g(\Div{Y}V, X) + g(Y, V)g(\Div{Z}V, X) \bigr] - 2\bigl[ g(X, V)g(Y, \Div{Z}V) \bigr],
           \end{aligned}
        \]
        最后两行用中括号括起的部分关于 $Y$, $Z$ 是对称的. 所以终于得到
        \[
            \begin{aligned}
            -\tDiv{Z}\psi(X, Y) + \tDiv{Y}\psi(X, Z) ={}& g(\Div{Y}\Div{Z}V, X) - g(\Div{\Div{Y}Z}V, X) - g(\Div{Z}\Div{Y}V, X) + g(\Div{\Div{Z}Y}V, X) \\
            ={}& \Rm(X, V, Y, Z) = -\Rm(V, X, Y, Z).
            \end{aligned}
        \]

        代入 (\ref{eq::1}) 式, 得到
        \[
            \tilde{C}(X, Y, Z) - C(X, Y, Z) = A \owedge g(V, X, Y, Z) - \Rm(V, X, Y, Z) = -W(V, X, Y, Z) = W(V, X, Z, Y).
        \]
        得证.
    \end{enumerate}
\end{solution}

\subsection{问题 1.4}
\begin{problem}
    考虑如下 Riemann 流形 $(\real^2, \dif s^2)$,
    \[
        \dif s^2 = \dfrac{\dif x^2 + \dif y^2}{1 + x^2 + y^2}.
    \]
    \begin{enumerate}
        \item 证明: 在无穷远处 $(\real^2, \dif s^2)$ 渐进于半径为 $1$ 的圆柱面, 即: $r \rightarrow \infty$ 时, $\gamma_r(t) = (r \cos t, r \sin t)$ 关于 $\dif s^2$ 的弧长趋向于 $2\pi$.
        \item 证明: $\Ric + \nabla^2 f = 0$, 其中 $f = -\log(1 + x^2 + y^2)$.
    \end{enumerate}
\end{problem}
\begin{solution}
    \begin{enumerate}
        \item 切向量长度为
        \[
            |\dot{\gamma_r}(t)|^2 = \dfrac{r^2}{1 + x^2 + y^2} = \dfrac{r^2}{1 + r^2}, 
        \]
        所以
        \[
            L(\gamma_r) = \int_{0}^{2\pi}\sqrt{\dfrac{r^2}{1 + r^2}} \dif t = 2\pi \sqrt{\dfrac{r^2}{1 + r^2}},
        \]
        所以 $r \rightarrow \infty$ 时弧长趋向于 $2\pi$.
        \item 利用共形变换的公式, 有
        \[
            \Ric = -\Delta (\dfrac{1}{2}f)g.
        \]
        显然有 $f_{,xx} = f_{,yy}$, 所以只要 $f_{,xy} = 0$, 就有 $\nabla^2 f = \dfrac{1}{2}\Delta f g$, 即 $\Ric + \nabla^2 f = 0$. 而
        \[
            \nabla^2 f(\partial_x, \partial_y) = \partial_y \partial_x f - \Div{\partial_y}\partial_x f.
        \]
        以及
        \[
            \Div{\partial_y}\partial_x = \partial_x(\dfrac{f}{2})\partial_y + \partial_y(\dfrac{f}{2})\partial_x = -\dfrac{x\partial_y + y\partial_x}{1 + x^2 + y^2}.
        \]
        得到
        \[
            \nabla^2 f(\partial_x, \partial_y) = \partial_y\bigl( \dfrac{-2x}{1 + x^2 + y^2} \bigr) + \dfrac{x\dfrac{-2y}{1 + x^2 + y^2} + y\dfrac{-2x}{1 + x^2 + y^2}}{1 + x^2 + y^2} = \dfrac{4xy}{(1 + x^2 + y^2)^2} - \dfrac{4xy}{(1 + x^2 + y^2)^2} = 0.
        \]
        因此 $\Ric + \nabla^2 f = 0$.
    \end{enumerate}
\end{solution}
\end{document}