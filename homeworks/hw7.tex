\documentclass{ctexart}

\usepackage{amsmath}
\usepackage{amssymb}
\usepackage{amsthm}
\usepackage{enumitem}
\usepackage{hyperref}

\usepackage{mathrsfs} % for \VecFld
\usepackage{stmaryrd}

\usepackage{color, framed}
\definecolor{shadecolor}{RGB}{241, 241, 255}
\newcounter{problemname}
\newenvironment{problem}{\begin{shaded}\stepcounter{problemname}\par\noindent\textbf{题目\arabic{problemname}. }}{\end{shaded}\par}
\newenvironment{solution}{\par\noindent\textbf{解答. }}{\par}

\newcommand{\cred}{\textcolor{red}}
\newcommand{\cblue}{\textcolor{blue}}
\newcommand{\ccyan}{\textcolor{cyan}}
\newcommand{\cmag}{\textcolor{magenta}}

% \usepackage{mathrsfs}

\usepackage{geometry}
\geometry{a4paper,scale=0.8}

% number fields
\newcommand{\real}{\mathbb{R}}
\newcommand{\Zahl}{\mathbb{Z}}
\newcommand{\half}{\mathbb{H}}

% function spaces
\newcommand{\cont}[1]{C^{#1}}

% manifold spaces
\newcommand{\sphere}{\mathbb{S}}        % a big S

% unitary operators
\def\dif{\mathop{}\!\mathrm{d}}         % differential
\DeclareMathOperator{\tr}{tr}           % trace
\newcommand{\Inner}[2]{\langle {#1}, {#2} \rangle}            % inner product
\newcommand{\norm}[1]{\Vert {#1} \Vert}

% differential manifold
\newcommand{\VecFld}{\mathscr{X}}
\newcommand{\vol}{\mathrm{vol}}

% Riemannian connection
\newcommand{\Rm}{\mathrm{Rm}}           % curvature tensor
\newcommand{\R}{\mathrm{R}}             % curvature tensor, sometimes scalar curvature
\newcommand{\K}{\mathrm{K}}             % sectional curvature
\newcommand{\Ric}{\mathrm{Ric}}         % Ricci tensor
\newcommand{\scal}{\mathrm{scal}}       % scalar curvature

\newcommand{\Div}[1]{\nabla_{#1}}

% submanifold
\newcommand{\bnabla}{\overline{\nabla}}
\newcommand{\tnabla}{\widetilde{\nabla}}
\newcommand{\bRm}{\overline{\Rm}}
\newcommand{\tRm}{\widetilde{\Rm}}

\newcommand{\bDiv}[1]{\bnabla_{#1}}
\newcommand{\tDiv}[1]{\tnabla_{#1}}

\newcommand{\Shape}[1]{A_{#1}}
\newcommand{\Secd}{\mathbf{h}}
\newcommand{\Second}[2]{\mathbf{h}(#1, #2)}
\newcommand{\MeanCV}{\mathbf{H}}

\newcommand{\pdDif}[2]{\dfrac{\partial #1}{\partial #2}}

\newcommand{\dDif}[2]{\dfrac{\dif #1}{\dif #2}}



\title{第七次作业}
\author{洪艺中 12335025}
\begin{document}
\maketitle
\subsection{149 页 问题 15}
\begin{problem}
    设 $ \sphere^m = \bigl\{ x \in \real^{m + 1} : \sum_{i} (x^i)^2 = r^2, r > 0 \bigr\} $. 作球极投影
    \[
        \phi \colon \sphere^m(r) \setminus \{(0, \cdots, 0, r)\} \rightarrow \real^m.
    \]
    证明: $\phi$ 为共形映射, 即对于 Riemann 流形 $(\sphere^m(r), \tilde{g}$ 和 $(\real^m, g)$, 有 $\tilde{g} = \phi^{\ast}g$, 这里 $\tilde{g}$ 是 $\sphere^m(r) \subset \real^{m + 1}$ 的诱导度量, $g$ 为 $\real^m$ 上的欧氏度量.
\end{problem}
\begin{solution}
    设 $\phi$ 将 $\sphere^{m}$ 投影到 $\real^{m + 1}$ 的 $x^{n + 1} = 0$ 平面, 也就是 $\real^m$. 设 $\real^m$ 的坐标是 $(y_1, \cdots, y_m)$, 那么 $\phi$ 可以表达为
    \[
        \phi(y_1, \cdots, y_n) = \biggl( \dfrac{2r^2y_1}{|y|^2 + r^2}, \cdots, \dfrac{2r^2y_m}{|y|^2 + r^2}, r\dfrac{|y|^2 - r^2}{|y|^2 + r^2} \biggr),
    \]
    计算切映射,
    \newcommand{\rightpart}[1]{#1}
    \[
        \phi^{\ast} = \dfrac{2r^2}{|y|^2 + r^2}
        \begin{bmatrix}
            \mathrm{I}_{m} & 0
        \end{bmatrix}
        - \dfrac{4 r^2}{(|y|^2 + r^2)^2}
        \begin{bmatrix}
            \rightpart{y_1^2} & \rightpart{y_1y_2} & \cdots & \rightpart{y_1y_m} & -ry_1 \\
            \rightpart{y_2y_1} & \rightpart{y_2^2} & \cdots & \rightpart{y_2y_m} & -ry_2 \\
            \cdots & \cdots & \ddots & \cdots & \vdots \\
            \rightpart{y_my_1} & \rightpart{y_my_2} & \cdots & \rightpart{y_m^2} & -ry_m
        \end{bmatrix}.
    \]
    容易发现, 若 $i \not= j$, 那么 $\Inner{\partial_{y_i}}{\partial_{y_j}}_{\tilde{g}} = 0$, 这是因为内积为
    \[
        \begin{aligned}
             {} & \Inner{\partial_{y_i}}{\partial_{y_j}}_{\tilde{g}} \\
            ={} & -\dfrac{2r^2}{|y|^2 + r^2} \cdot \dfrac{4 r^2}{(|y|^2 + r^2)^2} (y_iy_j + y_jy_i) + \biggl(\dfrac{4 r^2}{(|y|^2 + r^2)^2} \biggr)^2 \sum_{k = 1}^{m} \left( y_k y_i \cdot y_k y_j \right) + r^2 y_i y_j \\
            ={} & \dfrac{2r^2}{|y|^2 + r^2} \cdot \dfrac{4 r^2}{(|y|^2 + r^2)^2}(-2y_i y_j + 2y_i y_j) \\
            ={} & 0.
        \end{aligned}
    \]
    因此 $\tilde{g}$ 是对角的. 
    \[
        \begin{aligned}
             {} & \Inner{\partial_{y_i}}{\partial_{y_i}}_{\tilde{g}} \\
            ={} & \biggl( \dfrac{2r^2}{|y|^2 + r^2} \biggr)^2 - \dfrac{2r^2}{|y|^2 + r^2} \cdot \dfrac{4 r^2}{(|y|^2 + r^2)^2} 2y_i^2
            + \biggl( \dfrac{4 r^2}{(|y|^2 + r^2)^2} \biggr)^2 (\sum_{k = 1}^{m} \left( (y_k)^2 (y_i)^2 \right) - r^2 (y_i)^2) \\
            ={} & \dfrac{2r^2}{|y|^2 + r^2} \cdot \dfrac{4 r^2}{(|y|^2 + r^2)^2} \biggl( \dfrac{1}{2}(|y|^2 + r^2) - 2(y^i)^2 + 2(y^i)^2 \biggr) \\
            ={} & \dfrac{4r^4}{(|y|^2 + r^2)^2}.
        \end{aligned}
    \]
    即 $\tilde{g} = \dfrac{4r^4}{(|y|^2 + r^2)^2}g$, 所以 $\tilde{g}$ 是共形变换.
\end{solution}
\end{document}